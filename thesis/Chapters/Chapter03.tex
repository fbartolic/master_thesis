%************************************************
\chapter{Resonant passage of the 6:1 mean motion resonance}
\label{ch:analytical_model_of_high_order_mmrs}
%************************************************
After reviewing the theory required for studying
resonant capture in the framework of a three--body problem
with arbitrary mass ratios, we turn to the construction of
a one dimensional Hamiltonian for a particular mean motion 
resonance relevant to circumbinary planets. The question is,
which resonances are relevant?

Looking at the circumbinary planets orbiting main--sequence stars,
we see that all of them are located outside of the $5:1$ MMR
with the stellar binary. This is due to the fact that the inner
regions in period space are  mostly unstable because of resonance overlap
as will be shown in \cref{ch:numerical_analysis}. Thus, as the 
stellar binary evolves and the period ratio $P_o/P_i=n_i/n_o$ grows,
the first major resonance that will be encountered is the $6:1$ resonance,
which is 5th order.
More distant resonances such as $7:1$ and $8:1$ might be important
as well, however, since resonance 'strength' drops off with resonance
order, their effects will be less important. 
\section{The 6:1 mean motion resonance}
\label{sec:6_by_1_resonance}
A $6:1$ MMR is defined by the labels $n'=6,\,n=1$ in the disturbing
function expansion in \cref{eq:disturbing_function}. \Cref{tab:6_1_angles}
lists all of the harmonic angles associated with the principal
harmonics of a $6:1$ MMR together with the leading order of the
harmonic coefficient in the semi-major axis ratio $\alpha$ and
the inner and outer eccentricities. As expected, the sum of the
exponent powers of $e_i$ and $e_o$ is equal to 5 which is the order
of the resonance.
\begin{table}[h!]
\centering
\begin{tabular}{l c}
\toprule
    Harmonic angle $\phi_{mnn'}$ & Leading order of $\mathcal{R}_{mnn'}$\\
\midrule
    $\phi_{116}=\lambda_i-6\lambda_o+5\varpi_o$ & 
    $\mathcal{O}(\alpha^3,\,e_i^0,\,e_o^5)$\\
$\phi_{216}=\lambda_i-6\lambda_o+\varpi_i + 4\varpi_o$ & 
    $\mathcal{O}(\alpha^2,\,e_i^1,\,e_o^4)$\\
$\phi_{316}=\lambda_i-6\lambda_o+2\varpi_i + 3\varpi_o$ & 
    $\mathcal{O}(\alpha^3,\,e_i^2,\,e_o^3)$\\
$\phi_{416}=\lambda_i-6\lambda_o+3\varpi_i + 2\varpi_o$ & 
    $\mathcal{O}(\alpha^4,\,e_i^3,\,e_o^2)$\\
$\phi_{516}=\lambda_i-6\lambda_o+4\varpi_i + \varpi_o$ & 
    $\mathcal{O}(\alpha^5,\,e_i^4,\,e_o^1)$\\
$\phi_{616}=\lambda_i-6\lambda_o+5\varpi_i  $ & 
    $\mathcal{O}(\alpha^6,\,e_i^5,\,e_o^5)$\\
\bottomrule
\end{tabular}
\caption{Harmonic angles associated with the principal harmonics
    of a $6:1$ mean motion resonance (those with $n\leq m\leq n'$).}
\label{tab:6_1_angles}
\end{table}

In general we would have to consider all of the angles in \cref{tab:6_1_angles}
because near a $6:1$ commensurability they will all librate and it is not possible
to remove them via a canonical transformation because because of the small
divisor problem described in \cref{sec:small_divisor}.
There is however a single harmonic angle associated with the dominant term, the one with 
$m=2$, its harmonic coefficient is proportional to $\alpha^2$
while all the other harmonic coefficients
are higher order in $\alpha$. The difference between the dominant term and the 
next one is of order $\mathcal{O}(\alpha)$ in absolute value. For a $6:1$ resonance, 
$\alpha=a_i/a_o\approx 6^{-2/3}=0.3$, discarding the other terms is thus
not an ideal approximation but it is the one we have to make in order to get to 
an integrable Hamiltonian. 

After isolating only the $\mathcal{R}=\mathcal{R}_{216}
\cos\phi_{216}$ term in the disturbing function, the Hamiltonian
\ref{eq:three_body_hamiltonian} becomes
\begin{equation}
    \mathcal{H}&=\mathcal{H}_k+\mathcal{R}
    \label{eq:hamiltonian_orbital_elements}
\end{equation}
where
\begin{equation}
    \mathcal{H}_k&=\mathcal{H}_i+\mathcal{H}_o=
    -G \frac{m_1m_2}{a_i} -G \frac{m_{12}m_3}{a_o}
\end{equation}
is the Keplerian part of the Hamiltonian which depends only on the 
semi-major axes and
\begin{equation}
    \mathcal{R}=\frac{3}{4}\frac{G\mu_im_3}{a_o}\left(\frac{a_i}{a_o}\right)^2
    X^{2,2}_1(e_i)\,X^{-3,2}_6(e_o)\cos(\lambda_i-6\lambda_o+
    \varpi_i + 4\varpi_o)
\end{equation}
is the single resonant distrubing function term. Hamiltonian 
\ref{eq:hamiltonian_orbital_elements} is written in terms of the
orbital elements $(\lambda,a,e,\varpi)$ which do not form a canonically
conjugate set of variables. Proceed to rewrite the Hamiltonian in
so called \emph{Poincaré variables} which are often used in Celestial 
mechanics and do form a canonically conjugate set of variables. The Poincaré
variables are defined in terms of the orbital elements as
\begin{equation}
    \begin{aligned}
        \lambda_i&=\lambda_i &\Lambda_i&=\mu_i\sqrt{Gm_{12}a_i\\
        \gamma_i&=-\omega_i & \Gamma_i&=\mu_i\sqrt{Gm_{12}a_i}
    \left(1-\sqrt{1-e_i^2}\right)
\end{aligned}
\label{eq:poincare_variables}
\end{equation}
and similary for the outer orbit with $m_{12}\rightarrow m_{123}$ and the
index $o$ for the orbital elements. Remembering \cref{eq:ang_momentum} for 
the angular momentum of a Keplerian orbit, we see that the Poincaré 
Lambda-s correspond to the angular momentum for a circular outer and inner orbit
respectively. The Gammas are then the differences between two-body angular
momenta for a circular and elliptical orbit\footnote{In secular interactions
the sum of all $\Gamma$ elements of the system
is called the \emph{angular momentum deficit} (AMD). It is a conserved quantity
because the semi-major axes are constant and the total angular momentum is conserved.
\cite{laskar} has shown that the Solar System is AMD unstable in the sense that
if say all planets except, say, Venus attained maximum angular momentum (corresponding
to a circular orbit), Venus's eccentricity would inrease enough for crossing
orbits to occur.}
We can then solve the system 
\ref{eq:poincare_variables} for the orbital elements in terms of Poincaré
variables, the result is
\begin{equation}
    \begin{aligned}
        a_i&= \frac{\Lambda_i^2}{G\mu_i^2m_{12}} \\
        e_i &= \frac{1}{\Lambda_i} \sqrt{\Lambda_i^2-(\Gamma_i-\Lambda_i)^2}\\
        \varpi_i&=-\gamma_i
    \end{aligned}
    \label{eq:orbital_elems_in_terms_of_poincare}
\end{equation}
and again similarly for the outer orbital elements. We have taken the positive root
of $e_i$ since eccentricity is defined to be positive. The Hamiltonian
\ref{eq:hamiltonian_orbital_elements} expressed in terms of the new variables
is then
\begin{equation}
    \begin{aligned}
        \mathcal{H}&=-G^2 \frac{\mu_i^3m_{12}^2}{2\Lambda_i^2}  
        -G^2 \frac{\mu_o^3m_{123}^2}{2\Lambda_o^2}\\ 
        &-\frac{3}{4}G^2 \frac{\mu_o^6}{\mu_i^3} \frac{m_{123}^3m_3}{m_{12}^2}
        \frac{1}{\Lambda_o^2} \left(\frac{\Lambda_i}{\Lambda_o}\right)^4
    X^{2,2}_1(\Lambda_i)\,X^{-3,2}_6(\Lambda_o)\cos(\lambda_i-6\lambda_o
    -\gamma_i - 4\gamma_o)
    \end{aligned}
    \label{eq:hamiltonian_poincare}
\end{equation}
We can simplify the Hamiltonian \ref{eq:hamiltonian_poincare} by changing
to dimensionless units. This is achieved by scaling all masses, lengths and
time by a constant factor, as follows
\begin{equation}
    \hat m= \frac{m}{m'} \quad\hat a= \frac{a}{a'}\quad \hat t = \frac{t}{t'}  
\end{equation}
where $m$ stands for any quantity with the dimension of mass in 
\cref{eq:hamiltonian_poincare} and $a$ stands for any quantity with the 
dimension of length. Time is not present explicitely in \cref{eq:hamiltonian_poincare}.
The hats denote the fact that the new variables are dimensionless. Plugging in the
rescaled variables in the Hamiltonian (and taking out $G$, which has dimensions, from
the definition of the Poincaré momenta) we obtain a Hamiltonian of the form
\begin{equation}
    \mathcal{H}= \frac{Gm'^2}{a'} \mathcal{\hat H}
\end{equation}
where $\mathcal{\hat H}$ is now dimensionless and exactly the same as
\cref{eq:hamiltonian_poincare} except with $G$ factored out and all 
variables with a physical dimension given hats. The factor $Gm'^2/a'$ 
has dimensions of energy (as it should) and we can multiply the 
Hamiltonian $\mathcal{H}$ by its inverse to obtain a dimensionless 
Hamiltonian. Multiplying any Hamiltonian by a constant factor is 
equivalent to rescaling the time by that same factor\footnote{
    This follows from Hamilton's equations. Since 
    $\frac{\partial}{\partial p} (a
    \mathcal{H})= \frac{dq}{dt}$ where $a$ is some 
    constant, it follows that
$ \frac{\partial\mathcal{H}}{\partial p} = \frac{dq}{d(at)}$.}.
We then choose the scaling factors which give the simplest Hamiltonian,
a natural choice is $m'=m_{12}$ as a unit of mass, $\tilde{a}_i$ (the
reason for the use
of $\tilde$ will become clear in the next paragraph) as a unit of 
semi-major axis, and $1/n_i$ as a unit of time. Thus we can set
everywhere $m_{12}=\tilde{a}_i=\tilde{a}_o=1$. The dimensionless
Hamiltonian is then given by
\begin{equation}
    \begin{aligned}
        \mathcal{H}&=-\frac{\mu_i^3}{2\Lambda_i^2}  
        -\frac{\mu_o^3}{2\Lambda_o^2}\\ 
        &-\frac{3}{4} \frac{\mu_o^6}{\mu_i^3} 
       \frac{m_3}{\Lambda_o^2} \left(\frac{\Lambda_i}{\Lambda_o}\right)^4
    X^{2,2}_1(\Gamma_i)\,X^{-3,2}_6(\Gamma_o)\cos(\lambda_i-6\lambda_o
    -\gamma_i - 4\gamma_o)
    \end{aligned}
    \label{eq:hamiltonian_poincare_dimensionless}
\end{equation}
where we have omitted all the hats for clarity and we have used the 
approximation $m_{123}\approx m_{12}$, since the most massive planets
we will be considering are $m_3\approx 10^{-3} m_{12}$ which is negligible.

In order to reduce the Hamiltonian \ref{eq:hamiltonian_poincare_dimensionless},
to a form which resembles that of the pendulum, we have to expand it in a 
Taylor series around a location of an exact resonance. Since we are interested
in the dynamics in the vicinity of the resonance, this is a valid expansion.
We choose to expand the Keplerian part to \emph{second order} and the
resonant term to \emph{zeroth order} around $\Lambda_i=\tilde{\Lambda}_i$ 
and $\Lambda_o=\tilde{\Lambda}_o$, where 'tilde' denotes Poincaré
momenta evaluated at exact resonance\footnote{There is a subtlety  here concerning the
definition of $\Gamma_i$ and $\Gamma_o$ which is worth pointing out. After expanding the
resonant term about the resonance location to zeroth order in 
$\tilde{\Lambda}_i$ and $\tilde{\Lambda}_o$, we also choose to neglect
the variation of $a_i$ and $a_o$ in $\Gamma_i$ and $\Gamma_o$
because it is negligible in the Keplerian term. 
We thus have $\Gamma_i=\tilde{\Lambda}_i\left(1-\sqrt{1-e_i^2}\right)$ and
$\Gamma_o=\tilde{\Lambda}_o\left(1-\sqrt{1-e_o^2}\right)$.}
\begin{equation}
    \begin{align}
        \tilde{\Lambda}_i&=m_1m_2\sqrt{\tilde{a}_i/\tilde{a}_i}=m_1m_2=\mu_i\\
        \tilde{\Lambda}_o&=\mu_o\sqrt{\tilde{a}_o/\tilde{a}_i}=6^{1/3}\mu_o
    \end{align}
\end{equation}
where we have used Kepler's third law to evaluate the outer semi-major axis
at the location of $6:1$ MMR. The Hamiltonian 
\ref{eq:hamiltonian_poincare_dimensionless} becomes
\begin{equation}
    \begin{aligned}
        \mathcal{H}&=\frac{\mu_i^3}{2\tilde{\Lambda}_i^3}
        (\Lambda_i-\tilde{\Lambda}_i) - \frac{3}{2}
        \frac{\mu_i^3}{\tilde{\Lambda}_i^4} (\Lambda_i-\tilde{\Lambda}_i)^2+
    \frac{\mu_o^3}{2\tilde{\Lambda}_o^3}
        (\Lambda_o-\tilde{\Lambda}_o) - \frac{3}{2}
        \frac{\mu_o^3}{\tilde{\Lambda}_o^4} (\Lambda_o-\tilde{\Lambda}_o)^2\\
        &-\frac{3}{4} \frac{\mu_o^6}{\mu_i^3} 
        \frac{m_3}{\tilde{\Lambda}_o^2} \left(\frac{\tilde{\Lambda}_i}
        {\tilde{\Lambda}_o}\right)^4
    X^{2,2}_1(\Gamma_i)\,X^{-3,2}_6(\Gamma_o)\cos(\lambda_i-6\lambda_o
    -\gamma_i - 4\gamma_o)
    \end{aligned}
    \label{eq:hamiltonian_poincare_dimensionless_expanded}
\end{equation}
where we have ignored constant terms because Hamilton's equations are invariant
to an addition of a constant to the Hamiltonian. We then define new momenta
$J_i=\Lambda_i-\tilde{\Lambda}_i$ and $J_o=\Lambda_o-\tilde{\Lambda}_o$ 
which are shifted from $\Lambda_i$ and $\lambda_o$ by a constant (it is easy 
to check that the canonical form is preserved).

Finally, we turn to the Hansen coefficients which we have kept in symbolic 
form so far. There are no closed-form solutions for the integrals 
\cref{eq:hansen_inner,eq:hansen_outer} which define the coefficients. 
It is however possible to obtain a series solution in eccentricities 
using a computer algebra system. 
\Cite{Mardling2013} provides \texttt{Wolfram Mathematica} \citep{Mathematica} code 
for a series solution in eccentricities. The integral
for the inner Hansen coefficient is given by \cref{eq:hansen_inner} as
\begin{equation}
    \begin{aligned}
        X^{2,2}_1(e_i)= \frac{1}{2\pi} \int^{2\pi}_0 &(1-e_i\cos E_i)^{l+1}
        \left[\cos(mf_i)+i\sin(mf_i)\right]\\
        &\left[\cos(nM_i)-i\sin(nM_i)\right]\mathrm{d}E_i
    \end{aligned}
\end{equation}
where we have used the relation $(r/a_i)=1-e_i\cos E_i$ and wrote the 
exponentials in complex number form by using the Euler's identity. We 
do the same with $X^{-3,2}_6(e_o)$. Next, we replace the trigonometric 
terms and the mean anomaly
by using \cref{eq:eccentric_anomaly,kepler_equation}. Finally,
we expand the integrand in the inner coefficient in a Taylor series 
around $e_i$, and similarly around $e_o$ for the outer coefficient.

We can only use the lowest order approximation for the Hansen coefficinets
accurate to order 
$\mathcal{O}(e_i^3,e_o^6)$, otherwise the Hamiltonian becomes too complicated
later on. In order to establish the domain of validity of this approximation,
we plot the Hansen coefficients as functions of eccentricity.
\cref{fig:hansen_coefficients} shows that the first order approximation is 
valid up to about $e_i\leq 3$ for the inner eccentricity, and $e_o\leq 0.4$
for the outer. This is a fairly limiting assumption, however, the vast majority
of the observe circumbinary systems are comfortably in the low eccentricity
regime so it remains justified.
\begin{figure}[htb]
\centering
\includegraphics[width=\linewidth]{gfx/hansen_coefficients.pdf}
    \caption{Lowest order expansion term for the Hansen coefficients (dashed curves) 
    compared to exact integration.}
\label{fig:hansen_coefficients}
\end{figure}
Finally, after converting the series approximation to Poincaré momenta, the 
Hamiltonian \ref{eq:hamiltonian_poincare_dimensionless} becomes
\begin{equation}
    \begin{aligned}
        \mathcal{H}&=\frac{\mu_i^3}{2\tilde{\Lambda}_i^3}
        J_i - \frac{3}{2}
        \frac{\mu_i^3}{\tilde{\Lambda}_i^4} J_i^2+
    \frac{\mu_o^3}{2\tilde{\Lambda}_o^3}+
        J_o - \frac{3}{2}
        \frac{\mu_o^3}{\tilde{\Lambda}_o^4} J_o^2\\
        &+\frac{4797\sqrt{2}}{16}m_3 \frac{\mu_o^6}{\mu_i^3} 
        \frac{\tilde{\Lambda}_i^{ \frac{7}{2} }}
        {\tilde{\Lambda}_o^8}\sqrt{\Gamma_i}\Gamma_o^2
    \cos(\lambda_i-6\lambda_o+
    -\gamma_i - 4\gamma_o)
    \end{aligned}
    \label{eq:hamiltonian_poincare_dimensionless_expand}
\end{equation}

\section{Reduction to a single degree of freedom}
\label{sec:Reduction_to_a_sdof}
The Hamiltonian \ref{eq:hamiltonian_poincare_dimensionless_expanded} has four
degrees of freedom. We would like to find a canonical transformation which
reduces the number of degrees of freedom to one. It is known from one of the 
first Hamiltonian models of resonance \citep{Henrard1983} that a suitable 
canonical transformation has the harmonic angle as a position coordinate.
Inspired by this fact, we choose a canonical transformation to coordinates
$(\theta_1,\theta_2,\theta_3,\theta_4;\Theta_1,\Theta_2,\Theta_3,\Theta_4)$ 
generated by
\begin{equation}
    F_2= -(\lambda_i-6\lambda_o -\gamma_i - 4\gamma_o)\Theta_1 + 
    \lambda_i\Theta_2+ \lambda_o\Theta_3+\gamma_i\Theta_4
\end{equation}
From \cref{tab:generating_functions}, it follows that 
\begin{equation}
    \begin{aligned}
        J_1&= \frac{\partial F_2}{\partial \lambda_i}=-\Theta_1+\Theta_2 & 
        \quad\theta_1&= \frac{\partial F_2}{\partial\Theta_1}=
        - (\lambda_i-6\lambda_o -\gamma_i - 4\gamma_o)\\
J_2&= \frac{\partial F_2}{\partial \lambda_o}=6\Theta_1+\Theta_3 & 
        \quad\theta_2&= \frac{\partial F_2}{\partial\Theta_2}=\lambda_i\\
\Gamma_i&= \frac{\partial F_2}{\partial \gamma_i}=\Theta_1+\Theta_4& 
        \quad\theta_3&= \frac{\partial F_2}{\partial\Theta_3}=\lambda_o\\
\Gamma_o&= \frac{\partial F_2}{\partial \gamma_o}=4\Theta_1& 
        \quad\theta_4&= \frac{\partial F_2}{\partial\Theta_4}=\gamma_i\\
    \end{aligned}
    \label{eq:new_momenta}
\end{equation}
We can then easily solve for the new momenta in terms of old momenta
\begin{equation}
    \begin{align}
        \Theta_1&= \frac{1}{4} \Gamma_o\\
        \Theta_2&=J_1+ \frac{1}{4} \Gamma_o\\
        \Theta_3&=J_2 - \frac{3}{2} \Gamma_o\\
        \Theta_4&= \Gamma_i - \frac{1}{4} \Gamma_o
    \end{align}
\end{equation}
and the Hamiltonian expressed in terms of the new coordinates is 
\begin{equation}
    \begin{aligned}
        \mathcal{H}&=\left(- \frac{3\cdot 6^{2/3}}{2m_3} - \frac{3}{2\mu_i} 
        \right)\Theta^2+
        \left(\frac{3\Theta_2 }{\mu_i}
          -\sqrt[3]{\frac{9}{2}}\frac{\Theta_3}{m_3}
        \right)\Theta\\ 
        &+ \frac{533\cdot 2^{5/6}\sqrt[3]{3}}{24} 
        \frac{\sqrt{\mu_i}}{m_3}\;  \Theta^2\sqrt{\Theta+\Theta_4}\cos(\theta)
    \end{aligned}
    \label{eq:hamiltonian_sdof_full}
\end{equation}
Where we have taken $\theta_1\equiv\theta$ and $\Theta_1\equiv\Theta$.
The resulting Hamiltonian depends only on one coordinate $\theta$ with
$\Theta$ its momentum conjugate and 
is therefore a fully integrable single degree of freedom Hamiltonian. 
From Hamilton's equations, it follows that 
$\dot{\Theta}_2=\dot{\Theta}_3=\dot{\Theta}_4=0$, that is, $\Theta_1$,
$\Theta_2$  and $\Theta_3$ are constants of motion. 

The Hamiltonian \ref{eq:hamiltonian_sdof_full} depends on many parameters
which are constants, we wish to reduce the number of parameters to a 
smallest set of linearly independent parameters. We proceed by rewriting
\cref{eq:hamiltonian_sdof_full} as
\begin{equation}
    \mathcal{H}=\alpha\Theta^2+\beta\Theta +\epsilon\Theta^2\sqrt{\Theta+\Theta_4}
    \cos\theta
    \label{eq:hamiltonian_not_scaled}
\end{equation}
where
\begin{equation}
    \begin{aligned}
        \alpha&= - \frac{3\cdot 6^{2/3}}{2m_3} - \frac{3}{2\mu_i}\\
        \beta&=\frac{3\Theta_2 }{\mu_i}
          -\sqrt[3]{\frac{9}{2}}\frac{\Theta_3}{m_3} \\
         \epsilon&=\frac{533\cdot 2^{5/6}\sqrt[3]{3}}{24} 
        \frac{\sqrt{\mu_i}}{m_3}    
    \end{aligned}
\end{equation}
$\alpha$ and $\epsilon$ depend purely on the mass ratios $\mu_i=m_1m_2$ and
$m_3$, the $\epsilon$ parameter depends on $\Theta_2$ and $\Theta_3$ which in
turn depend on the distance to the resonance. In order to further reduce the 
number of parameters, we scale the momentum $\Theta$ by means of a 
simple scale transformation $\Theta\rightarrow\eta\Theta$, where
$\eta$ is a constant factor to be determined;
remembering that we also have to scale time by the same factor
. Hamiltonian \ref{eq:hamiltonian_not_scaled} becomes
\begin{equation}
    \mathcal{H}=\eta^2\alpha\Theta^2+\eta\beta\Theta +
    \eta^{5/2}\epsilon\Theta^2\sqrt{\Theta+ \frac{\Theta_4}{\eta}}
    \cos\theta)
\end{equation}
We then choose the scaling parameter $\eta$ such that the coefficients
in front of the first and the last term become equal, that is,
we require that 
\begin{equation}
    \eta^2\alpha=\eta^{5/2}\epsilon
\end{equation}
It follows that 
\begin{equation}
    \eta=\left( \frac{\alpha}{\epsilon} \right)^2
\end{equation}
and the Hamiltonian is
  \begin{equation}
    \mathcal{H}= \frac{\alpha^5}{\epsilon^4} \Theta^2+
      \beta \frac{\alpha^2}{\epsilon^2} \Theta +
    \frac{\alpha^5}{\epsilon^4} \Theta^2\sqrt{\Theta+ \frac{\Theta_4}{\eta}}
    \cos\theta
\end{equation}
We can now multiply the hamiltonian by the dimensionless factor 
$\frac{\epsilon^4}{\alpha^5}$ which corresponds to rescaling the time
again. The final Hamiltonian then has the form
\begin{equation}
    \mathcal{H}= \Theta^2-\delta\Theta +
     \Theta^2\sqrt{\Theta+ c}
    \cos\theta
    \label{eq:hamiltonian_final}
\end{equation}
where the two constants $\delta$ and $c$ are given by
\begin{align}
    \delta&= -\frac{\beta\epsilon^2}{\alpha^3}\label{eq:delta}\\
    c&= \left( \frac{\epsilon}{\alpha} \right)^2 \Theta_4
    =\left( \frac{\epsilon}{\alpha} \right)^2\left(\Gamma_i- \frac{1}{4}
    \Gamma_o\right)\label{eq:c}
\end{align}
We have thus managed to reduce the $6:1$ resonant Hamiltonian to the
simplest possible form, a one degree of freedom Hamiltonian with two
parameters. We now turn to studying its structure.
\section{The resonance structure}
\label{sec:The resonance structure}
\begin{figure}[htb]
\centering
    \includegraphics[width=\linewidth]{gfx/ham_params.pdf}
    \caption{(a) Hamiltonian parameter $\delta$
     on the period ratio $\mu_i=0.5$, $m_3=10^{-3}$, $e_i=0.2$, $e_o=0.05$.
     (b)Hamiltonian parameter $c$ as a function of the inner
     eccentricity for $\mu_i=0.5$, $m_3=10^{-3}$, $e_o=0.05$.}
\label{fig:ham_params}
\end{figure}
In order to gain insight into the dependance of the two parameters $\delta$
and $c$, we plot them in \cref{fig:ham_params}. We see that $\delta$ is a 
measure of proximity to the exact commensurability since it reaches 
zero at $P_o/P_i\approx 6$. In the case of an evolving circumbinary 
system $\delta$ monotonically decreases from a positive value to a negative
value or equivalently, a changing period ratio can be seen as varying
$\delta$. Thus, we can model the passage through the resonance through 
$\delta$.  

The $c$ parameter by definition stays constant as the period ratio varies
because the Poincaré momenta $\Gamma_i$ and $\Gamma_o$ in \cref{eq:c}
are evaluated at exact resonance. More interesting is the dependence
of $c$ on the inner eccentricity, shown in panel (b) of \cref{fig:ham_params}.
We see that $c>0$ for all $e_i$ for a particular value of $e_o$. In fact, the
plot shown in (b) stays almost exactly the same for all reasonable values
of $e_o$. Negative values of $c$ occur only for extremely small eccentricities
which never really occur because some eccentricity is always induced due to 
secular interactions. In what follows, we therefore consider only the case
$c>0$.

The fixed points of the Hamiltonian are given by Hamilton's equations
as
\begin{equation}
    \frac{\partial \mathcal{H}}{\partial\Theta} = 
    \frac{\partial\mathcal{H}}{\partial\theta}=0
\end{equation}
We obtain 
\begin{align}
    -\Theta^2\sqrt{\Theta +c}\sin\theta&=0\\
     \frac{\Theta^2}{2\sqrt{\Theta + c}}\cos\theta +2\Theta\sqrt{\Theta+c}
    +2\Theta-\delta&=0
    \label{eq:saddle_point}
\end{align}
The only non trivial solution for the first equation is $\theta=\{0,\pi\}$
for $\theta\in\{0,2\pi\}$. The second equation then becomes
\begin{equation}
    (-1)^s\frac{\Theta^2}{2\sqrt{\Theta + c}} +2\Theta\sqrt{\Theta+c}
    2\Theta-\delta=0
    \label{eq:fixed_points}
\end{equation}
where $s=\{0,1\}$. There is no analytic solution for \cref{eq:fixed_points}, 
we can gain insight into the possible solution by searching for the roots
graphically. We define $R=\sqrt{2\Theta}$ and rewrite \cref{eq:fixed_points}
as
\begin{equation}
    f(\delta, R)=g(R,c)
\end{equation}
where
\begin{equation}
    f(\delta,R) = -\delta+R^2
\end{equation}
and
\begin{equation}
    g(R,c)=(-1)^s \frac{  R^4}{8\sqrt{ \frac{1}{2} R^2+c}} 
    +R^2\sqrt{ \frac{1}{2}R^2+c} 
\end{equation}
\begin{figure}[htb]
\centering
\includegraphics[width=0.9\linewidth]{gfx/fixed_points.pdf}
\caption{A graphical solution to the equation defining the fixed points of
    the Hamiltonian \ref{eq:hamiltonian_final}. The solid lines are 
    plots of $f(\delta,R)$ for various values of $\delta$. The dashed
    lines are plots of the function $g(R, c)$  for a fixed positive value
    of $c=0.1$.}
\label{fig:fixed_points}
\end{figure}
\Cref{fig:fixed_points} shows a plot of $f(\delta,R)$ and $g(R,c)$
for a fixed value of $c$. Each solid line corresponds to a different
value of delta. The intersections of $f$ and $g$ are the roots of 
\cref{eq:fixed_points}. There are two solutions for $\delta>0$
and no solutions for $\delta<0$. At $\delta=0$ and $\Phi=0$ a double root
of \cref{eq:fixed_points} either appears or disappears depending on wheater
zero is approached from above or below. Thus, we conclude that there are two 
distinct behaviours of the Hamiltonian depending on the sign of $\delta$,
$\delta=0$ is then called a \emph{bifurcation point} of the
Hamiltonian \ref{eq:hamiltonian_final}.
\footnote{In the theory of dynamical systems, a bifurcation occurs when a small 
change in a certain parameter of a system causes a sudden \emph{topological} 
change in its behaviour.}

The properties of Hamiltonian \ref{eq:hamiltonian_final} are most easily seen
in phase space plots for different values of $\delta$. Instead of $(\Theta,\theta)$ 
coordinates, we plot the Hamiltonian
in so called Poincaré Cartesian variables, defined as 
\begin{equation}
    \begin{aligned}
        x&=\sqrt{2\Theta}\cos\theta\\
        y&=\sqrt{2\Theta}\sin\theta
    \end{aligned}
\end{equation}
This would be a standard polar to Cartesian transformation were it not for the 
square roots. The square roots are necessary if the transformation is to be 
canonical \citep{sylvio}, and the factor of 2 is for convenience. The reason
we use these coordinates is because the coordinates $(\theta,\Theta)$ are
singular at the origin $\Theta=0$ since $\theta$ becomes ill defined. 
The Hamiltonian in the new coordinates is then
\begin{equation}
    \mathcal{H}= \frac{1}{4}\left( x^2+y^2\right) - \frac{1}{2}\delta\left(
    x^2+y^2\right)+ \frac{1}{8} x\left(x^2+y^2\right)^{3/2}\sqrt{4c+2x^2+2y^2}
    \label{eq:hamiltonian_cartesian}
\end{equation}
Because the coordinates $(\theta,\Theta)$ are ill-defined at the orgin, when
solving \cref{eq:fixed_points} for the fixed points, we missed a third fixed 
point at the origin. It is easy to see by writing down the Hamilton's
equations in $(x,y)$ coordinates that the origin $(0,0)$ is always a fixed
point, independent of the value of $\delta$.
\begin{figure}
\centering
\includegraphics[width=0.8\linewidth]{gfx/phase_space_plot.pdf}
    \caption{Phase space portraits of the Hamiltonian 
    \ref{eq:hamiltonian_cartesian} for different values of 
    $\delta$. The red curve
    is the separatrix. Filled circles are stable fixed points and the open
    circle is the unstable saddle point.}
\label{fig:phase_space}
\end{figure}

The phase space structure is shown in \cref{fig:phase_space} for three different
values of $\delta$, consider for now the first panel which shows the phase space
structure for positive $\delta$. Each curve is a constant trajectory defined by a specific value
of $\mathcal{H}$. Given specific values of $\delta$, and $c$, the Hamiltonian 
\ref{eq:hamiltonian_final} is completely defined and the system moves around on
one of the level curves in \cref{fig:phase_space} for all time. The radial distance
from the origin to a point on any given level curve is proportional to the eccentricity
of the outer orbit since $\Theta\propto
\Gamma_o\propto e_o^2$. There are three fixed points, the
leftmost one and the origin are stable and the right most one is unstable. The 
stability of a fixed point is easily calculated by evaluating the Hessian matrix in
\cref{eq:fixed_points_stability}, but is also apparent from the figure because there
appear to be no small amplitude circulatory trajectories around the rightmost point. 

The phase space of the first two panels of \cref{fig:phase_space} is divided into two
\emph{topologically}\footnote{Topology is a field in mathematics which studies 
properties of space preserved under continuous deformations such as stretching, twisting
and crumpling. For example, a circle is topologically equivalent to an ellipse and 
a sphere is equal to an ellipsoid, and a coffe cup is equivalent to a torus.} different
regions divided by the separatrix curve (red curve). The separatrix curve passes through
the unstable saddle point and consists of two
'branches', an inner branch (small red quasi-circle) and on outer branch (large red
quasi-circle).  The two branches intersect the $x$ axis at two points, the region between
them is called the resonant region. The trajectories 
in between the two branches are called \emph{librations} and they do not include the origin,
the trajectories outside the outer branch and inside the inner branch of the separatrix
are called \emph{circulations} and they do include the origin. The sytem is said to be
in resonance if its trajectory is a curve within the resonant region, in which case $\theta$
does not pass through all the values in $\{0,2\pi\}$ and is slowly varying. Depending on the
distance from the resonance center (the leftmost stable fixed point), a resonant trajectory
can include large excursions in eccentricity.

If we vary the $\delta$ parameter from a positive value (first panel of \cref{fig:phase_space}),
to a negative value the phase space changes. As $\delta$ decreases towards zero the two
outer fixed points approach the origin (second panel in \cref{fig:phase_space}), finally at 
$\delta=0$, corresponding to an exact $6:1$ commensurability, the three fixed points coalesce
into a single fixed point at the origin. For negative $\delta$ there are only circulatory 
trajectories enclosing the origin and no resonant trajectories are possible. This changing
behaviour of the Hamiltonian based on the sign of $\delta$ is best illustrated using
a \emph{bifurcation digram}, shown in \cref{fig:morphogenesis}. The bifurcation diagram
shows the value of the Hamiltonian at the two outer fixed points as a function of $\delta$.
The dashed curve curve is the value of $\mathcal{H}$ evaluated at the saddle point, and the 
solid line is the value of $\mathcal{H}$ at the left center point. The possible 
types of trajectories are indicated in the diagram.
\begin{figure}[htb]
\centering
\includegraphics[width=\linewidth]{gfx/morphogenesis.pdf}
    \caption{The bifurcation diagram of the the Hamiltonian \ref{eq:hamiltonian_final}. 
    The dashed curve denotes the value of $\mathcal{H}$ evaluated at the saddle point $x_s$, and 
    the solid curve the value of $\mathcal{H}$ at the left center point.}
\label{fig:morphogenesis}
\end{figure}

While the analysis beforehand has concentrated on a particular $6:1$ MMR, it is 
worth mentioning 
that many of the properties of the phase space (number of fixed points, shape of
separatrix) look similar for other 
resonances, most major differences are in the structure of the bifurcation diagram.

\section{Resonance passage}
\label{sec:Resonance_passage}
Finally, we turn to the topic of resonance passage. As mentioned previously, we
can model the resonance passage as a slowly varying $\delta$ parameter in Hamiltonian 
\ref{eq:hamiltonian_final}. 
Looking at \cref{fig:phase_space}, an immediate consequence of the resonant
passage is visible. If the system starts at positive delta and ends up at
negative delta, the resonant angle is necessarily circulating at the end. 
Thus, divergent
resonance passage ($\dot{\delta}<0$) cannot result in a capture into resonance.
Since the two orbits in an evolving circumbinary system
are diverging away from each other, the $\delta$ parameter starts positive 
($P_o/P_i<6$), passes through zero and end up being negative ($P_o/P_i>6$).
As long as the stellar binary's orbit decays due to tidal forces, no capture 
will occur in such systems.

The assumption that $\delta$ varies \emph{slowly} is crucial. As long as
$\dot{\delta}/\delta$ is small compared to the libration or circulation 
frequency,or equivalently the characteristic time of resonance passage
is large compared to the libration/circulation period, the system 
approximately follows the level curves in \cref{fig:phase_space}. In 
other words, we require 
\begin{equation}
    \bigg\lvert\frac{\dot{\delta}}{\delta}\bigg\rvert \ll \frac{1}{T} 
    \label{eq:adabatic_criterion}
\end{equation}
where $T$ is the period of circulatory or libratory motion.
In reality, an evolving circumbinary system is fully 
described only by a nonconservative Hamiltonian because the tidal decay
forces acting on the stellar binary dissipate energy. Such Hamiltonians
are however analytically intractable.

One can show that given a Hamiltonian $\mathcal{H}(\theta,\Theta;\delta)$
where $\delta$ is slowly varying, there exists a constant of motion
called the \emph{adiabatic invariant} (see for example \cite{landau}),
given by
\begin{equation}
    \mathcal{J}=\oint \Theta\,\mathrm{d}\theta    
    \label{eq:adiabatic_invariant}
\end{equation}
where $\oint$ means that the integration is done over the complete range
of variation of the coordinate variable, in our case, $\{0,2\pi\}$. Along
such  path the Hamiltonian and $\delta$ are constant. The integral
\ref{eq:adiabatic_invariant} is in fact the definition of the famous
physical \emph{action} from the \emph{principle of least action}, thus, 
is an adiabatic invariant.

$\mathcal{J}$
remains constant on average when $\delta$ is varied and it has a 
particularly simple interpretation in the case of a one dimensional
Hamiltonian, namely, it is equal to the area enclosed by a level
curve of the Hamiltonian in $(x,y)$ coordinates. For a circular trajectory
\begin{equation}
    \mathcal{J}_\text{circ}=2\pi\,\Theta=\pi R^2
    \label{eq:J_circular}
\end{equation}
Remember that since $\mathcal{J}\propto\Theta$, it depends on the 
eccentricity of the outer orbit. 

For a general trajectory, the integral in \cref{eq:adiabatic_invariant}
is most easily evaluated by switching the integration variable to $\Theta$.
Using $\frac{\mathrm{d}\theta}{dt}=\frac{\partial\mathcal{H}}{\partial\Theta}$
and $ \frac{\mathrm{d}\Theta}{dt} =- \frac{\partial\mathcal{H}}{\partial\theta}$,
we have 
\begin{equation}
    \mathcal{J}=-\oint \frac{\frac{\partial\mathcal{H}}{\partial\Theta}}{
        \frac{\partial\mathcal{H}}{\partial\theta}}
    \bigg\rvert_{\mathcal{H}=\mathcal{H}_\text{curve}}\Theta\mathrm{d}\Theta 
\end{equation}
where $\mathcal{H}_\text{curve}$ is the value of the Hamiltonian corresponding
to a particular level curve.

There is caveat to the conservation of $\mathcal{J}$ however. If the trajectory 
approaches
the separatrix, the librational period tends to inifinity (remember the pendulum
model). As $T\rightarrow\infty$, the adiabatic criterion in 
\cref{eq:adabatic_criterion} is no longer satisfied and $\mathcal{J}$ is not
conserved. After passing the separatrix curve however, $\mathcal{J}$ is 
one again conserved. At the exact moment of separatrix crossing, the 
adiabatic invariant experiences a discontinuity, it is this discontinuity
which is responsible for a sudden increase in eccentricity in the case
of divergent resonance crossing.
\begin{figure}[htb]
\centering
\includegraphics[width=\linewidth]{gfx/phase_space_plot_adiabatic.pdf}
    \caption{The evolution of the adiabatic invariant $\mathcal{J}$
    with varying $\delta$.
    Dark circle denotes the adiabatic invariant, red curve is the 
    separatrix. Top left panel - initially the trajectory is a circle.
    Top right panel - at $\delta=0.0001$ the initial adiabatic invariant
    is equal to the area enclosed by the inner separatrix. Bottom left
    - the trajectory expands to the outer branch of the separatrix which
    corresponds to a discontinuity in $\mathcal{J}$. Bottom right -
    at $\delta=-0.0001$ the phase space has only circulatory trajectories
    and the adiabatic invariant is larger than initially.}
\label{fig:phase_space_adiabatic}
\end{figure}

\Cref{fig:phase_space_adiabatic} shows the evolution of the adiabatic 
invariant as $\delta$ varies. Assuming the system is not in resonance
to begin with, it starts on a circular trajectory with $\mathcal{J}
=2\pi\Theta_\text{init}$ within the inner branch of the separatrix
(top left panel, grey circle). At a particular $\delta=\delta_t$
whic depends on the initial area of the circle, the adiabatic 
invariant is equal to the area enclosed by the inner branch of the
separatrix (top right panel). The adiabatic invariant has no choice
but to suddenly expand to the value equal to the area of the
outer separatrix branch (bottom left panel). Finally, as $\delta$
decreases further and goes past $\delta=0$, the separatrix
disappears, the adiabatic invariant is again conserved and the 
system is left on a trajectory with a larger $\Theta$ (and
therefore larger $e_O$) then it inially started with.

In order to calculate the adiabatic invariant after the discontinuous
jump and therefore also the final eccentricity, we need to know
the initial adiabatic invariant (the initial eccentricity) and
the areas enclosed by the two branches of the separatrix for any
given $\delta$. We start by calculating the separatrix areas, 
plugging in the Hamilton's equations into \cref{eq:adiabatic_invariant},
we have
\begin{equation}
    \mathcal{J}=\oint \frac{1}{\Theta \sqrt{\Theta + c} 
    \sin{\left (\theta \right )}} 
    \left(\frac{\Theta^{2} \cos{\left (\theta \right )}}
    {2 \sqrt{\Theta + c}} + 2 \Theta \sqrt{\Theta + c} 
    \cos{\left (\theta \right )} + 2 \Theta - \delta\right)\mathrm{d}\,\Theta
    \label{eq:intermediate_integral}
\end{equation}
To solve the integral, we need an expression for $\theta$ as a
function of $\Theta$ on the level curve corresponding to the separatrix.
Since the separatrix passes through the saddle point, it is defined by
the equation
\begin{equation}
    \mathcal{H}_s=\Theta^2-\delta\Theta+\Theta^2\sqrt{\Theta+c}\cos\theta
    \label{eq:sepratrix_eq}
\end{equation}
where $\mathcal{H}_s$ is the value of the Hamiltonian at the saddle point.
We can easily solve \cref{eq:sepratrix_eq} for $\cos\theta$ and $\sin\theta$
using the trigonometric identity $\sin^2\theta+\cos^2\theta=1$. Plugging
those expressions into \cref{eq:intermediate_integral}, we have
\begin{equation}
    \mathcal{J}=2\oint_{\Theta_1}^{\Theta_2}\frac{\Theta 
    \left(\Theta + c\right) \left(2 \Theta - 
    \delta\right) + \frac{\Theta}{2} \left(- \Theta^{2} + \Theta \delta 
    + \mathcal{H}_s\right) + 2 \left(\Theta + c\right) \left(- \Theta^{2}
    + \Theta \delta + \mathcal{H}_s\right)}{\sqrt{\frac{1}{\Theta + c} 
    \left(\Theta^{4} \left(\Theta + c\right) - \left(- \Theta^{2} +
    \Theta \delta + \mathcal{H}_s\right)^{2}\right)} \left(\Theta + 
    c\right)^{\frac{3}{2}}}\mathrm{d}\Theta
    \label{eq:massive_integral}
\end{equation}
Where we have used the symmetry of the separatrix to integrate along 
its upper part. The lower limit of integration is the value the location of the saddle
point $\Theta_1=\Theta_s$, the upper limit is the value of $\Theta$
at $\theta=\pi$ where the inner and the outer branch cross the separatrix.

Since it is not possible to derive an analytic solution for the location
of the fixed points and hence also the poins at which the separatrix
crosses the $x$ axis, we have to resort to numerical methods of finding
the equation roots. To find the roots we use
the \texttt{Python} package \texttt{SciPy} \citep{scipy} and its
well-tested routine \texttt{fsolve} for finding roots of nonlinear equations.
The algorithm for finding the area enclosed by the separatrix is then
\begin{enumerate}
    \item Determine the location of the saddle point by finding the root
        for \cref{eq:saddle_point} with $\theta=0$. There is only one
        root and the algorithm easily converges.\\
    \item Evaluate the Hamiltonian \ref{eq:hamiltonian_final} at the 
        location of the saddle point denoting the value by $\mathcal{H}_s$
        , this determines the separatrix curve.\\
    \item Solve \cref{eq:sepratrix_eq} for the two roots at $\theta=\pi$ 
        which are the locations at which the two branches of the separatrix 
        cross the $x$ axis. The shape of the function is simple and the
        root-finding algorithm easily converges.\\
    \item Solve the integral \ref{eq:massive_integral} using the \texttt{SciPy}
        module \texttt{integrate.quad}. The lower limit of integration is the
        location of the saddle point and the upper is either the root 
        of the lower separatrix branch or the upper one, depending on which
        area we are interested in.
\end{enumerate}

Now that we have an algorithm which determines the areas enclosed by the two
branches of the separatrix, the eccentricity kick can be determined as
follows
\begin{enumerate}
    \item Calculate $\mathcal{J}_\text{init}=2\pi\,\Theta_\text{init}$ for given
        system parameters and an initial eccentricity $e_o$.\\
    \item We require $\mathcal{J}_\text{init}=\mathcal{J}_\text{inner}$ 
        where $\mathcal{J}_\text{inner}$ is the area enclosed by the inner
        branch of the separatrix. To find the value of $\delta=\delta_t$ for
        which this equality is true, we numerically solve for the root of
        $\mathcal{J}_\text{init}-\mathcal{J}_\text{inner}(\delta)=0$.
    \item The final action $\mathcal{J}_\text{final}$ is then equal to
        the area enclosed by outer branch of the separatrix evaluated at
        $\delta=\delta_t$, that is, $\mathcal{J}_\text{final}=
        \mathcal{J}_\text{outer}(\delta_t)$.
    \item Once the resonance passage is over the trajectory is again circulatory
        and we can recover the final eccentricity by solving the equation
        \begin{equation}
            \mathcal{J}_\text{final}=2\pi\,\Theta_\text{final}=
            \frac{\pi}{2} 6^{1/3}m_3\left(1-\sqrt{1-e_o^2}\right)
        \end{equation}
        for $e_o$. 
\end{enumerate}

