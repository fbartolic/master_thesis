%************************************************
\chapter{N-body simulations of circumbinary planets up to Common Envelope phase}
\label{ch:NBODY_simulations}
%************************************************
While analytic models can give important insight into the physics 
of divergent resonance passage in a circumbinary planetary system
they cannot compare to the full solution of the equations of motion.
Here I describe a different approach to the problem, by means of direct
N-body simulations coupled with simulations of the stellar binary. 
These simulations provide a clear picture of the dynamical evolution
of the system.

\section{The N-body problem}
\label{sec:The N-body problem}

\section{REBOUND - an open source N-body integrator}
\label{sec:REBOUND - an open source N-body integrator}
- mention integrators, what does order stand for
- symplectic vs non-symplectic integrators
- describe IAS15 in more detail
- mention reproducibility and simulation archive
- MEGNO as a criterion for stabilility (not sure this should be here) 

\section{binary\_c - a binary stellar evolution code}
\label{sec:binary_c - a binary stellar evolution code}
- describe what it does, mention it's built on BSE
- Python interface to the C library

\section{Stability of observed circumbinary planets on the main sequence}
\label{sec:Stability of observed circumbinary planets on the main sequence}
- MEGNO maps of Kepler planets together with resonance bubbles

\section{Stellar evolution trajectories}
\label{sec:Stellar evolution trajectories}


\section{Intial conditions}
\label{sec:Intial conditions}
