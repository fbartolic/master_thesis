%************************************************
\chapter{Resonant passage of the 6:1 mean motion resonance}
\label{ch:analytical_model_of_high_order_mmrs}
%************************************************
After reviewing the theory required for studying
resonant capture in the framework of a three--body problem
with arbitrary mass ratios, we turn to the construction of
a one dimensional Hamiltonian for a particular mean motion 
resonance relevant to circumbinary planets. The question is,
which resonances are relevant?

Looking at the circumbinary planets orbiting main--sequence stars,
we see that all of them are located outside of the $5:1$ MMR
with the stellar binary. This is due to the fact that the inner
regions in period space are  mostly unstable because of resonance overlap
as will be shown in \cref{ch:numerical_analysis}. Thus, as the 
stellar binary evolves and the period ratio $P_o/P_i=n_i/n_o$ grows,
the first major resonance that will be encountered is the $6:1$ resonance,
which is 5th order.
More distant resonances such as $7:1$ and $8:1$ might be important
as well, however, since resonance 'strength' drops off with resonance
order, their effects will be less important. 
\section{The 6:1 mean motion resonance}
\label{sec:6_by_1_resonance}
A $6:1$ MMR is defined by the labels $n'=6,\,n=1$ in the disturbing
function expansion in \cref{eq:disturbing_function}. \Cref{tab:6_1_angles}
lists all of the harmonic angles associated with the principal
harmonics of a $6:1$ MMR together with the leading order of the
harmonic coefficient in the semi-major axis ratio $\alpha$ and
the inner and outer eccentricities. As expected, the sum of the
exponent powers of $e_i$ and $e_o$ is equal to 5 which is the order
of the resonance.
\begin{table}[h!]
\centering
\begin{tabular}{l c}
\toprule
    Harmonic angle $\phi_{mnn'}$ & Leading order of $\mathcal{R}_{mnn'}$\\
\midrule
    $\phi_{161}=\lambda_i-6\lambda_o+5\varpi_o$ & 
    $\mathcal{O}(\alpha^3,\,e_i^0,\,e_o^5)$\\
$\phi_{261}=\lambda_i-6\lambda_o+\varpi_i + 4\varpi_o$ & 
    $\mathcal{O}(\alpha^2,\,e_i^1,\,e_o^4)$\\
$\phi_{361}=\lambda_i-6\lambda_o+2\varpi_i + 3\varpi_o$ & 
    $\mathcal{O}(\alpha^3,\,e_i^2,\,e_o^3)$\\
$\phi_{461}=\lambda_i-6\lambda_o+3\varpi_i + 2\varpi_o$ & 
    $\mathcal{O}(\alpha^4,\,e_i^3,\,e_o^2)$\\
$\phi_{561}=\lambda_i-6\lambda_o+4\varpi_i + \varpi_o$ & 
    $\mathcal{O}(\alpha^5,\,e_i^4,\,e_o^1)$\\
$\phi_{661}=\lambda_i-6\lambda_o+5\varpi_i  $ & 
    $\mathcal{O}(\alpha^6,\,e_i^5,\,e_o^5)$\\
\bottomrule
\end{tabular}
\caption{Harmonic angles associated with the principal harmonics
    of a $6:1$ mean motion resonance (those with $n\leq m\leq n'$).}
\label{tab:6_1_angles}
\end{table}

The Harmonic angle associated with the dominant term is the one with
$m=2$ because its harmonic coefficient is proportional to $\alpha^2$
while all the other harmonic coefficients associated with this MMR
are higher order in $\alpha$. We can now write down the Hamiltonian
in \cref{eq:three_body_hamiltonian} with $\mathcal{R}=\mathcal{R}_{216}
\cos\phi_{216}$.
\begin{equation}
    \mathcal{H}&=\mathcal{H}_k+\mathcal{R}
    \label{eq:hamiltonian_orbital_elements}
\end{equation}
where
\begin{equation}
    \mathcal{H}_k&=\mathcal{H}_i+\mathcal{H}_o=
    -G \frac{m_1m_2}{a_i} -G \frac{m_{12}m_3}{a_o}
\end{equation}
is the Keplerian part of the Hamiltonian which depends only on the 
semi-major axes and
\begin{equation}
    \mathcal{R}=\frac{3}{4}\frac{G\mu_im_3}{a_o}\left(\frac{a_i}{a_o}\right)^2
    X^{2,2}_1(e_i)\,X^{-3,2}_6(e_o)\cos(\lambda_i-6\lambda_o+
    \varpi_i + 4\varpi_o)
\end{equation}
is the single resonant distrubing function term. Hamiltonian 
\ref{eq:hamiltonian_orbital_elements} is written in terms of the
orbital elements $(\lambda,a,e,\varpi)$ which do not form a canonically
conjugate set of variables. Proceed to rewrite the Hamiltonian in
so called \emph{Poincaré variables} which are often used in Celestial 
mechanics and do form a canonically conjugate set of variables. The Poincaré
variables are defined in terms of the orbital elements as
\begin{equation}
    \begin{aligned}
        \lambda_i&=\lambda_i &\Lambda_i&=\mu_i\sqrt{Gm_{12}a_i\\
        \gamma_i&=-\omega_i & \Gamma_i&=\mu_i\sqrt{Gm_{12}a_i}
    \left(1-\sqrt{1-e_i^2}\right)
\end{aligned}
\label{eq:poincare_variables}
\end{equation}
and similary for the outer orbit with $m_{12}\rightarrow m_{123}$ and the
index $o$ for the orbital elements. Remembering \cref{eq:ang_momentum} for 
the angular momentum of a Keplerian orbit, we see that the Poincaré 
Lambda-s correspond to the angular momentum for a circular outer and inner orbit
respectively. The Gammas are then the differences between two-body angular
momenta for a circular and elliptical orbit\footnote{In secular interactions
the sum of all $\Gamma$ elements of the system
is called the \emph{angular momentum deficit} (AMD). It is a conserved quantity
because the semi-major axes are constant and the total angular momentum is conserved.
\cite{laskar} has shown that the Solar System is AMD unstable in the sense that
if say all planets except, say, Venus attained maximum angular momentum (corresponding
to a circular orbit), Venus's eccentricity would inrease enough for crossing
orbits to occur.}
We can then solve the system 
\ref{eq:poincare_variables} for the orbital elements in terms of Poincaré
variables, the result is
\begin{equation}
    \begin{aligned}
        a_i&= \frac{\Lambda_i^2}{G\mu_i^2m_{12}} \\
        e_i &= \frac{1}{\Lambda_i} \sqrt{\Lambda_i^2-(\Gamma_i-\Lambda_i)^2}\\
        \varpi_i&=-\gamma_i
    \end{aligned}
    \label{eq:orbital_elems_in_terms_of_poincare}
\end{equation}
and again similarly for the outer orbital elements. We have taken the positive root
of $e_i$ since eccentricity is defined to be positive. The Hamiltonian
\ref{eq:hamiltonian_orbital_elements} expressed in terms of the new variables
is then
\begin{equation}
    \begin{aligned}
        \mathcal{H}&=-G^2 \frac{\mu_i^3m_{12}^2}{2\Lambda_i^2}  
        -G^2 \frac{\mu_o^3m_{123}^2}{2\Lambda_o^2}\\ 
        &-\frac{3}{4}G^2 \frac{\mu_o^6}{\mu_i^3} \frac{m_{123}^3m_3}{m_{12}^2}
        \frac{1}{\Lambda_o^2} \left(\frac{\Lambda_i}{\Lambda_o}\right)^4
    X^{2,2}_1(\Lambda_i)\,X^{-3,2}_6(\Lambda_o)\cos(\lambda_i-6\lambda_o
    -\gamma_i - 4\gamma_o)
    \end{aligned}
    \label{eq:hamiltonian_poincare}
\end{equation}
We can simplify the Hamiltonian \ref{eq:hamiltonian_poincare} by changing
to dimensionless units. This is achieved by scaling all masses, lengths and
time by a constant factor, as follows
\begin{equation}
    \hat m= \frac{m}{m'} \quad\hat a= \frac{a}{a'}\quad \hat t = \frac{t}{t'}  
\end{equation}
where $m$ stands for any quantity with the dimension of mass in 
\cref{eq:hamiltonian_poincare} and $a$ stands for any quantity with the 
dimension of length. Time is not present explicitely in \cref{eq:hamiltonian_poincare}.
The hats denote the fact that the new variables are dimensionless. Plugging in the
rescaled variables in the Hamiltonian (and taking out $G$, which has dimensions, from
the definition of the Poincaré momenta) we obtain a Hamiltonian of the form
\begin{equation}
    \mathcal{H}= \frac{Gm'^2}{a'} \mathcal{\hat H}
\end{equation}
where $\mathcal{\hat H}$ is now dimensionless and exactly the same as
\cref{eq:hamiltonian_poincare} except with $G$ factored out and all 
variables with a physical dimension given hats. The factor $Gm'^2/a'$ 
has dimensions of energy (as it should) and we can multiply the 
Hamiltonian $\mathcal{H}$ by its inverse to obtain a dimensionless 
Hamiltonian. Multiplying any Hamiltonian by a constant factor is 
equivalent to rescaling the time by that same factor\footnote{
    This follows from Hamilton's equations. Since 
    $\frac{\partial}{\partial p} (a
    \mathcal{H})= \frac{dq}{dt}$ where $a$ is some 
    constant, it follows that
$ \frac{\partial\mathcal{H}}{\partial p} = \frac{dq}{d(at)}$.}.
We then choose the scaling factors which give the simplest Hamiltonian,
a natural choice is $m'=m_{12}$ as a unit of mass, $\tilde{a}_i$ (the
reason for the use
of $\tilde$ will become clear in the next paragraph) as a unit of 
semi-major axis, and $1/n_i$ as a unit of time. The dimensionless
Hamiltonian is then given by
\begin{equation}
    \begin{aligned}
        \mathcal{H}&=-\frac{\mu_i^3}{2\Lambda_i^2}  
        -\frac{\mu_o^3}{2\Lambda_o^2}\\ 
        &-\frac{3}{4} \frac{\mu_o^6}{\mu_i^3} 
       \frac{m_3}{\Lambda_o^2} \left(\frac{\Lambda_i}{\Lambda_o}\right)^4
    X^{2,2}_1(\Gamma_i)\,X^{-3,2}_6(\Gamma_o)\cos(\lambda_i-6\lambda_o
    -\gamma_i - 4\gamma_o)
    \end{aligned}
    \label{eq:hamiltonian_poincare_dimensionless}
\end{equation}
where we have omitted all the hats for clarity and we have used the 
approximation $m_{123}\approx m_{12}$, since the most massive planets
we will be considering are $m_3\approx 10^{-3} m_{12}$ which is negligible.

In order to reduce the Hamiltonian \ref{eq:hamiltonian_poincare_dimensionless},
to a form which resembles that of the pendulum, we have to expand it in a 
Taylor series around a location of an exact resonance. Since we're interested
in the dynamics in the vicinity of the resonance, this is a valid expansion.
We choose to expand the Keplerian part to \emph{second order} and the
resonant term to \emph{zeroth order} around $\Lambda_i=\tilde{\Lambda}_i$ 
and $\Lambda_o=\tilde{\Lambda}_o$, where 'tilde' denotes Poincaré
momenta evaluated at exact resonance
\begin{equation}
    \begin{align}
        \tilde{\Lambda}_i&=m_1m_2\sqrt{\tilde{a}_i/\tilde{a}_i}=m_1m_2\\
        \tilde{\Lambda}_o&=\mu_o\sqrt{\tilde{a}_o/\tilde{a}_i}=6^{1/3}\mu_o
    \end{align}
\end{equation}
where we have used Kepler's third law to evaluate the outer semi-major axis
at the location of $6:1$ MMR. The Hamiltonian 
\ref{eq:hamiltonian_poincare_dimensionless} becomes
\begin{equation}
    \begin{aligned}
        \mathcal{H}&=\frac{\mu_i^3}{2\tilde{\Lambda}_i^3}
        (\Lambda_i-\tilde{\Lambda}_i) - \frac{3}{2}
        \frac{\mu_i^3}{\tilde{\Lambda}_i^4} (\Lambda_i-\tilde{\Lambda}_i)^2
    \frac{\mu_o^3}{2\tilde{\Lambda}_o^3}
        (\Lambda_o-\tilde{\Lambda}_o) - \frac{3}{2}
        \frac{\mu_o^3}{\tilde{\Lambda}_o^4} (\Lambda_o-\tilde{\Lambda}_o)^2\\
        &-\frac{3}{4} \frac{\mu_o^6}{\mu_i^3} 
        \frac{m_3}{\tilde{\Lambda}_o^2} \left(\frac{\tilde{\Lambda}_i}
        {\tilde{\Lambda}_o}\right)^4
    X^{2,2}_1(\Gamma_i)\,X^{-3,2}_6(\Gamma_o)\cos(\lambda_i-6\lambda_o
    -\gamma_i - 4\gamma_o)
    \end{aligned}
    \label{eq:hamiltonian_poincare_dimensionless_expanded}
\end{equation}
where we have ignored constant terms because Hamilton's equations are invariant
to an addition of a constant to the Hamiltonian. We then define new momenta
$J_i=\Lambda_i-\tilde{\Lambda}_i$ and $J_o=\Lambda_o-\tilde{\Lambda}_o$ 
which are shifted from $\Lambda_i$ and $\lambda_o$ by a constant (it is easy 
to check that the canonical form is preserved).
\section{Reduction to a single degree of freedom}
\label{sec:Reduction_to_a_sdof}
The Hamiltonian \ref{eq:hamiltonian_poincare_dimensionless_expanded} has four
degrees of freedom. We would like to find a canonical transformation which
reduces the number of degrees of freedom to one. It is known from one of the 
first Hamiltonian models of resonance \citep{Henrard1983} that a suitable 
canonical transformation has the harmonic angle as a position coordinate.
Inspired by this fact, we choose a canonical transformation  to coordinates
$(\theta_1,\theta_2,\theta_3,\theta_4;\Theta_1,\Theta_2,\Theta_3,\Theta_4)$ 
generated by
\begin{equation}
    F_2= -(\lambda_i-6\lambda_o -\gamma_i - 4\gamma_o)\Theta_1 + 
    \lambda_i\Theta_2+ \lambda_o\Theta_3+\gamma_i\Theta_4
\end{equation}
From \cref{tab:generating_functions}, it follows that 
\begin{equation}
    \begin{aligned}
        J_1&= \frac{\partial F_2}{\partial \lambda_i}=-\Theta_1+\Theta_2 & 
        \quad\theta_1&= \frac{\partial F_2}{\partial\Theta_1}=
        - (\lambda_i-6\lambda_o -\gamma_i - 4\gamma_o)\\
J_2&= \frac{\partial F_2}{\partial \lambda_o}=6\Theta_1+\Theta_3 & 
        \quad\theta_2&= \frac{\partial F_2}{\partial\Theta_2}=\lambda_i\\
\Gamma_i&= \frac{\partial F_2}{\partial \gamma_i}=\Theta_1+\Theta_4& 
        \quad\theta_3&= \frac{\partial F_2}{\partial\Theta_3}=\gamma_o\\
\Gamma_o&= \frac{\partial F_2}{\partial \gamma_o}=4\Theta_1& 
        \quad\theta_4&= \frac{\partial F_2}{\partial\Theta_4}=\gamma_i\\
    \end{aligned}
    \label{eq:new_momenta}
\end{equation}
We can then easily solve for the new momenta in terms of all momenta
\begin{equation}
    \begin{align}
        \Theta_1&= \frac{1}{4} \Gamma_o\\
        \Theta_2&=J_1+ \frac{1}{4} \Gamma_o\\
        \Theta_3&=J_2 - \frac{3}{2} \Gamma_o\\
        \Theta_4&= \Gamma_i - \frac{1}{4} \Gamma_o
    \end{align}
\end{equation}
and the Hamiltonian expressed in terms of the new coordinates is 
\begin{equation}
    \begin{aligned}
        \mathcal{H}&=\left(- \frac{1}{2} \frac{\mu_i^3}{\tilde{\Lambda}_i^3}
         +3 \frac{\mu_o^3}{\tilde{\Lambda}_o^3}+3\Theta_2 \frac{\mu_i^3}
         {\tilde{\Lambda}_i^4} -18\Theta_3\frac{\mu_o^3}{\tilde{\Lambda}_o^4} 
        \right)\Theta_1 +
         \left(-\frac{3}{2}\frac{\mu_i^3}{\tilde{\Lambda}_i^4} - 
         54\frac{\mu_o^3}{\tilde{\Lambda}_o^4}\right)\Theta_1^2\\
        &-\frac{3}{4} \frac{\mu_o^6}{\mu_i^3} 
        \frac{m_3}{\tilde{\Lambda}_o^2} \left(\frac{\tilde{\Lambda}_i}
        {\tilde{\Lambda}_o}\right)^4
        X^{2,2}_1(\Theta_1+\Theta_4)\,X^{-3,2}_6(\Theta_1)\cos(\theta_1)
    \end{aligned}
    \label{eq:hamiltonian_sdof_full}
\end{equation}
The resulting Hamiltonian depends only on one coordinate $\theta_1$ with
$\Theta_1$ its momentum conjugate and 
is therefore a single degree of freedom Hamiltonian. From Hamilton's
equations, it follows that $\dot{\Theta}_2=\dot{\Theta}_3=\dot{\Theta}_4=0$.

Finally, we turn to the Hansen coefficients which we have kept in symbolic 
form so far. There are no closed-form solutions for the integrals 
\cref{eq:hansen_inner,eq:hansen_outer} which define the coefficients. It is however
possible to obtain a series solution using a computer algebra system. 
\Cite{Mardling2013} provides \texttt{Wolfram Mathematica} \citep{Mathematica} code 
for a series solution in eccentricities. However, we are working in Poincaré 
coordinates and would like a solution as a series in $\Gamma_i$ and $\Gamma_o$. 
The integral
for the inner Hansen coefficient is given by \cref{eq:hansen_inner} as
\begin{equation}
    \begin{aligned}
        X^{2,2}_1(e_i)= \frac{1}{2\pi} \int^{2\pi}_0 &(1-e_i\cos E_i)^{l+1}
        \left[\cos(mf_i)+i\sin(mf_i)\right]\\
        &\left[\cos(nM_i)-i\sin(nM_i)\right]\mathrm{d}E_i
    \end{aligned}
\end{equation}
where we have used the relation $(r/a_i)=1-e_i\cos E_i$ and wrote the 
exponentials in complex number form by using the Euler's identity. We 
do the same with $X^{-3,2}_6(e_o)$. We eliminate $e_i$ using 
\cref{eq:new_momenta,eq:orbital_elems_in_terms_of_poincare}
\begin{equation}
    e_i = \frac{1}{\tilde{\Lambda}_i} \sqrt{\tilde{\Lambda}_i^2-
    (\Theta_1+\Theta_4-\tilde{\Lambda}_i)^2}
\end{equation}
and similarly for the outer eccentricity
    \begin{equation}
        e_o = \frac{1}{\tilde{\Lambda}_o} \sqrt{\tilde{\Lambda}_o^2
        -(4\Theta_1+-\tilde{\Lambda}_i)^2}
\end{equation}
Next, we replace the trigonometric terms and the mean anomaly
by using \cref{eq:eccentric_anomaly,kepler_equation}. Finally,
we exapand the integrand in a Taylor series around $\Theta_1=
\Gamma_o/4\approx m_3\sqrt{\tilde{a}_o}e_o^2/2$ which is a
small parameter.



We can only use the lowest order approximation accurate to order 
$\mathcal{O}(e_i^3,e_o^6)$, otherwise the Hamiltonian becomes too complicated
later on. In order to establish the domain of validity of this approximation,
we plot the Hansen coefficients as functions of eccentricity.
\cref{fig:hansen_coefficients} shows that the first order approximation is 
valid up to abou $e_i\leq 3$ for the inner eccentricity, and $e_o\leq 0.4$
for the outer. This is a fairly limiting assumption, however, the vast majority
of the observe circumbinary systems are comfortably in the low eccentricity
regime so it remains justified.
\begin{figure}[htb]
\centering
\includegraphics[width=\linewidth]{gfx/hansen_coefficients.pdf}
    \caption{Lowest order expansion term for the Hansen coefficients (dashed curves) 
    compared to exact integration.}
\label{fig:hansen_coefficients}
\end{figure}
Finally, after converting the series approximation to Poincaré momenta, the 
Hamiltonian \ref{eq:hamiltonian_poincare_dimensionless} becomes
\begin{equation}
    \begin{aligned}
        \mathcal{H}&=\frac{\mu_i^3}{2\tilde{\Lambda}_i^3}
        J_i - \frac{3}{2}
        \frac{\mu_i^3}{\tilde{\Lambda}_i^4} J_i^2
    \frac{\mu_o^3}{2\tilde{\Lambda}_o^3}
        J_o - \frac{3}{2}
        \frac{\mu_o^3}{\tilde{\Lambda}_o^4} J_o^2\\
        &+\frac{4797\sqrt{2}}{16} \frac{\mu_o^6}{\mu_i^3} 
        \frac{m_3}{\tilde{\Lambda}_o^2} \left(\frac{\tilde{\Lambda}_i}
        {\tilde{\Lambda}_o}\right)^4
    \cos(\lambda_i-6\lambda_o+
    -\gamma_i - 4\gamma_o)
    \end{aligned}
    \label{eq:hamiltonian_poincare_dimensionless}
\end{equation}



Where we have taken $\theta_1\equiv\theta$ and $\Theta_1\equiv\Theta$.
