%************************************************
\chapter{An analytical model of high-order mean--motion resonances}
\label{ch:analytical_model_of_high_order_mmrs}
%************************************************
The second chapter deals with analytic models of mean--motion resonances
(MMRs for short). This concept is the key part of the thesis as 
resonance overlap and resonance passage are crucial for the determination
of the stability of circumbinary planets. Here I develop a new analytical
model of high-order resonance in the case where the inner two bodies are 
comparable in mass, something which hasn't been done before. I use this 
model to predict the eccentricity kick to a planet orbiting evolving stars
as these stars come closer together due to tidal forces. 

\section{The two--body problem}
\label{sec:The two--body problem}
- be as brief as possible, describe mention runge-lenz vector
conservation and the fixed position of orbit in space


\section{A brief review of Hamiltonian mechanics}
\label{sec:A brief review of Hamiltonian mechanics}
- be as clear as possible
- HM as one of several equivalent formulations, don't forget about
orders of diff. equations and Hamilton-Jacobi formulation
- don't forget that the key point of Hamilton's formulation is 
the equivalence of positions and momenta and the freedom to transform
the coordinates as we please as long as the transformation is canonical
- define a degree of freedom
- define what it means for a system to be integrable
- define generating functions
- maybe Liouville theorem

\section{The pendulum}
\label{sec:The pendulum}
- follow either Mardling's non-hamiltonian approach or Allice's approach
- libration periods, driving forces...

\section{The three--body problem}
\label{sec:The three--body problem}
- follow Mardling's nbody lectures, introduce mean mean--motion
resonances mathematically as slowly varying angles in  Fourier series
and physically, as repeated conjuctions of planets at the same location
in space.
- KAM theorem and invariant torii
- disturbing function
- secular and resonant interactions
- resonance widths (no plots for now)
- show that any Hamiltonian with a particular Harmonic can be reduced to
a single degree of freedom
- show that it is possible to systematically remove all other harmonics
by means of a canonical transformation

\section{A model of the 6:1 resonance}
\label{sec:6_by_1_resonance}
- here goes almost everything from the Jupyter notebook

\section{7:1 and higher-order resonances}
\label{sec:7:1 and higher-order resonances}
