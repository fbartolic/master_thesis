%*******************************************************
% Abstract
%*******************************************************
%\renewcommand{\abstractname}{Abstract}
\pdfbookmark[1]{Abstract}{Abstract}
\begingroup
\let\clearpage\relax
\let\cleardoublepage\relax
\let\cleardoublepage\relax

\chapter*{Abstract}
Circumbinary planets have been observed around both main sequence and post 
common envelope binary stars. It is not clear if a circumbinary system on the
main sequence can survive the post main sequence evolution leading to a
post common envelope system. In this work we investigate the evolution 
of circumbinary planets after the binary stars leaves the main sequence and prior to
the onset of common envelope evolution. In particular, we focus on the role
of divergent mean motion resonance passage which can excite the eccentricity 
of the planet. We develop a Hamiltonian model of the previously not 
studied high order $6:1$ resonance, applicable to circumbinary systems 
where the mass ratio
between the primary and the secondary star can be large. We 
then integrate
the circumbinary system using an N-body code coupled with a stellar evolution 
code and compare the results with the analytical model. The results show that
the resonances have a small effect in most cases with the effect of secular
decrease in eccentricity being dominant. In the majority of 
cases the planets survive the 
evolution prior to the common envelope.
\vfill

\endgroup			

\vfill
