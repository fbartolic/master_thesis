%************************************************
\documentclass[ twoside,openright,titlepage,numbers=noenddot,headinclude,%1headlines,% letterpaper a4paper
                footinclude=true,cleardoublepage=empty,abstractoff, % <--- obsolete, remove (todo)
                BCOR=5mm,paper=a4,fontsize=11pt,%11pt,a4paper,%
                american,%
                ]{scrreprt}

%********************************************************************
% Note: Make all your adjustments in here
%*******************************************************
% ****************************************************************************************************
% classicthesis-config.tex 
% formerly known as loadpackages.sty, classicthesis-ldpkg.sty, and classicthesis-preamble.sty 
% Use it at the beginning of your ClassicThesis.tex, or as a LaTeX Preamble 
% in your ClassicThesis.{tex,lyx} with % ****************************************************************************************************
% classicthesis-config.tex 
% formerly known as loadpackages.sty, classicthesis-ldpkg.sty, and classicthesis-preamble.sty 
% Use it at the beginning of your ClassicThesis.tex, or as a LaTeX Preamble 
% in your ClassicThesis.{tex,lyx} with % ****************************************************************************************************
% classicthesis-config.tex 
% formerly known as loadpackages.sty, classicthesis-ldpkg.sty, and classicthesis-preamble.sty 
% Use it at the beginning of your ClassicThesis.tex, or as a LaTeX Preamble 
% in your ClassicThesis.{tex,lyx} with \input{classicthesis-config}
% ****************************************************************************************************  
% If you like the classicthesis, then I would appreciate a postcard. 
% My address can be found in the file ClassicThesis.pdf. A collection 
% of the postcards I received so far is available online at 
% http://postcards.miede.de
% ****************************************************************************************************


% ****************************************************************************************************
% 0. Set the encoding of your files. UTF-8 is the only sensible encoding nowadays. If you can't read
% äöüßáéçèê∂åëæƒÏ€ then change the encoding setting in your editor, not the line below. If your editor
% does not support utf8 use another editor!
% ****************************************************************************************************
\PassOptionsToPackage{utf8}{inputenc}
	\usepackage{inputenc}

% ****************************************************************************************************
% 1. Configure classicthesis for your needs here, e.g., remove "drafting" below 
% in order to deactivate the time-stamp on the pages
% ****************************************************************************************************
\PassOptionsToPackage{eulerchapternumbers,listings,drafting,%
					 pdfspacing,%floatperchapter,%linedheaders,%
					 subfig,beramono,eulermath,parts}{classicthesis}                                        
% ********************************************************************
% Available options for classicthesis.sty 
% (see ClassicThesis.pdf for more information):
% drafting
% parts nochapters linedheaders
% eulerchapternumbers beramono eulermath pdfspacing minionprospacing
% tocaligned dottedtoc manychapters
% listings floatperchapter subfig
% ********************************************************************


% ****************************************************************************************************
% 2. Personal data and user ad-hoc commands
% ****************************************************************************************************
\newcommand{\myTitle}{Planets orbiting evolving binary stars\xspace}
\newcommand{\mySubtitle}{Subtitle\xspace}
%\newcommand{\myDegree}{Doktor-Ingenieur (Dr.-Ing.)\xspace}
\newcommand{\myName}{Fran Bartolić\xspace}
\newcommand{\myProf}{Put name here\xspace}
\newcommand{\myOtherProf}{Put name here\xspace}
\newcommand{\mySupervisor}{Put name here\xspace}
\newcommand{\myFaculty}{Put data here\xspace}
\newcommand{\myDepartment}{Put data here\xspace}
\newcommand{\myUni}{Department of Physics, University of Rijeka\xspace}
\newcommand{\myLocation}{Rijeka, Croatia\xspace}
\newcommand{\myTime}{July 2017\xspace}
%\newcommand{\myVersion}{version 4.2\xspace}

% ********************************************************************
% Setup, finetuning, and useful commands
% ********************************************************************
\newcounter{dummy} % necessary for correct hyperlinks (to index, bib, etc.)
\newlength{\abcd} % for ab..z string length calculation
\providecommand{\mLyX}{L\kern-.1667em\lower.25em\hbox{Y}\kern-.125emX\@}
\newcommand{\ie}{i.\,e.}
\newcommand{\Ie}{I.\,e.}
\newcommand{\eg}{e.\,g.}
\newcommand{\Eg}{E.\,g.} 
% ****************************************************************************************************


% ****************************************************************************************************
% 3. Loading some handy packages
% ****************************************************************************************************
% ******************************************************************** 
% Packages with options that might require adjustments
% ******************************************************************** 
%\PassOptionsToPackage{ngerman,american}{babel}   % change this to your language(s)
% Spanish languages need extra options in order to work with this template
%\PassOptionsToPackage{spanish,es-lcroman}{babel}
	\usepackage{babel}                  

\usepackage{csquotes}
\PassOptionsToPackage{%
    %backend=biber, %instead of bibtex
	backend=bibtex8,bibencoding=ascii,%
	language=auto,%
	style=numeric-comp,%
    %style=authoryear-comp, % Author 1999, 2010
    %bibstyle=authoryear,dashed=false, % dashed: substitute rep. author with ---
    sorting=nyt, % name, year, title
    maxbibnames=10, % default: 3, et al.
    %backref=true,%
    natbib=true % natbib compatibility mode (\citep and \citet still work)
}{biblatex}
    \usepackage{biblatex}

\PassOptionsToPackage{fleqn}{amsmath}       % math environments and more by the AMS 
    \usepackage{amsmath}

\usepackage{siunitx} % for physical units
\newcommand{\vect}[1]{\boldsymbol{\mathbf{#1}}} % shorthand for bold vectors
% ******************************************************************** 
% General useful packages
% ******************************************************************** 
\PassOptionsToPackage{T1}{fontenc} % T2A for cyrillics
    \usepackage{fontenc}     
\usepackage{textcomp} % fix warning with missing font shapes
\usepackage{scrhack} % fix warnings when using KOMA with listings package          
\usepackage{xspace} % to get the spacing after macros right  
\usepackage{mparhack} % get marginpar right
%\usepackage[latest]{latexrelease} % will be used once available in more distributions (ISSUE #107)
\PassOptionsToPackage{printonlyused,smaller}{acronym} 
    \usepackage{acronym} % nice macros for handling all acronyms in the thesis
    %\renewcommand{\bflabel}[1]{{#1}\hfill} % fix the list of acronyms --> no longer working
    %\renewcommand*{\acsfont}[1]{\textsc{#1}} 
    \renewcommand*{\aclabelfont}[1]{\acsfont{#1}}
% ****************************************************************************************************


% ****************************************************************************************************
% 4. Setup floats: tables, (sub)figures, and captions
% ****************************************************************************************************
%\usepackage{tabularx} % better tables
%    \setlength{\extrarowheight}{3pt} % increase table row height
\usepackage{booktabs} % better tables
\renewcommand{\arraystretch}{1.3} (or 1.3)
\newcommand{\tableheadline}[1]{\multicolumn{1}{c}{\spacedlowsmallcaps{#1}}}
\newcommand{\myfloatalign}{\centering} % to be used with each float for alignment
\usepackage{caption}
% Thanks to cgnieder and Claus Lahiri
% http://tex.stackexchange.com/questions/69349/spacedlowsmallcaps-in-caption-label
% [REMOVED DUE TO OTHER PROBLEMS, SEE ISSUE #82]    
%\DeclareCaptionLabelFormat{smallcaps}{\bothIfFirst{#1}{~}\MakeTextLowercase{\textsc{#2}}}
%\captionsetup{font=small,labelformat=smallcaps} % format=hang,
\captionsetup{font=small} % format=hang,
\usepackage{subfig}  
% ****************************************************************************************************


% ****************************************************************************************************
% 5. Setup code listings
% ****************************************************************************************************
\usepackage{listings} 
%\lstset{emph={trueIndex,root},emphstyle=\color{BlueViolet}}%\underbar} % for special keywords
\lstset{language=[LaTeX]Tex,%C++,
    morekeywords={PassOptionsToPackage,selectlanguage},
    keywordstyle=\color{RoyalBlue},%\bfseries,
    basicstyle=\small\ttfamily,
    %identifierstyle=\color{NavyBlue},
    commentstyle=\color{Green}\ttfamily,
    stringstyle=\rmfamily,
    numbers=none,%left,%
    numberstyle=\scriptsize,%\tiny
    stepnumber=5,
    numbersep=8pt,
    showstringspaces=false,
    breaklines=true,
    %frameround=ftff,
    %frame=single,
    belowcaptionskip=.75\baselineskip
    %frame=L
} 
% ****************************************************************************************************             


% ****************************************************************************************************
% 6. PDFLaTeX, hyperreferences and citation backreferences
% ****************************************************************************************************
% ********************************************************************
% Using PDFLaTeX
% ********************************************************************
\PassOptionsToPackage{pdftex,hyperfootnotes=false,pdfpagelabels}{hyperref}
    \usepackage{hyperref}  % backref linktocpage pagebackref
\pdfcompresslevel=9
\pdfadjustspacing=1 
\PassOptionsToPackage{pdftex}{graphicx}
    \usepackage{graphicx} 
 

% ********************************************************************
% Hyperreferences
% ********************************************************************
\hypersetup{%
    %draft, % = no hyperlinking at all (useful in b/w printouts)
    colorlinks=true, linktocpage=true, pdfstartpage=3, pdfstartview=FitV,%
    % uncomment the following line if you want to have black links (e.g., for printing)
    %colorlinks=false, linktocpage=false, pdfstartpage=3, pdfstartview=FitV, pdfborder={0 0 0},%
    breaklinks=true, pdfpagemode=UseNone, pageanchor=true, pdfpagemode=UseOutlines,%
    plainpages=false, bookmarksnumbered, bookmarksopen=true, bookmarksopenlevel=1,%
    hypertexnames=true, pdfhighlight=/O,%nesting=true,%frenchlinks,%
    urlcolor=webbrown, linkcolor=RoyalBlue, citecolor=webgreen, %pagecolor=RoyalBlue,%
    %urlcolor=Black, linkcolor=Black, citecolor=Black, %pagecolor=Black,%
    pdftitle={\myTitle},%
    pdfauthor={\textcopyright\ \myName, \myUni, \myFaculty},%
    pdfsubject={},%
    pdfkeywords={},%
    pdfcreator={pdfLaTeX},%
    pdfproducer={LaTeX with hyperref and classicthesis}%
}   
\usepackage{cleveref}
% ********************************************************************
% Setup autoreferences
% ********************************************************************
% There are some issues regarding autorefnames
% http://www.ureader.de/msg/136221647.aspx
% http://www.tex.ac.uk/cgi-bin/texfaq2html?label=latexwords
% you have to redefine the makros for the 
% language you use, e.g., american, ngerman
% (as chosen when loading babel/AtBeginDocument)
% ********************************************************************
\makeatletter
\@ifpackageloaded{babel}%
    {%
       \addto\extrasamerican{%
			\renewcommand*{\figureautorefname}{Figure}%
			\renewcommand*{\tableautorefname}{Table}%
			\renewcommand*{\partautorefname}{Part}%
			\renewcommand*{\chapterautorefname}{Chapter}%
			\renewcommand*{\sectionautorefname}{Section}%
			\renewcommand*{\subsectionautorefname}{Section}%
			\renewcommand*{\subsubsectionautorefname}{Section}%     
                }%
           % Fix to getting autorefs for subfigures right (thanks to Belinda Vogt for changing the definition)
            \providecommand{\subfigureautorefname}{\figureautorefname}%             
    }{\relax}
\makeatother


% ****************************************************************************************************
% 7. Last calls before the bar closes
% ****************************************************************************************************
% ********************************************************************
% Development Stuff
% ********************************************************************
\listfiles
%\PassOptionsToPackage{l2tabu,orthodox,abort}{nag}
%   \usepackage{nag}
%\PassOptionsToPackage{warning, all}{onlyamsmath}
%   \usepackage{onlyamsmath}
% ********************************************************************
% Last, but not least...
% ********************************************************************
\usepackage{classicthesis} 
% ****************************************************************************************************


% ****************************************************************************************************
% 8. Further adjustments (experimental)
% ****************************************************************************************************
% ********************************************************************
% Changing the text area
% ********************************************************************
%\linespread{1.05} % a bit more for Palatino
%\areaset[current]{312pt}{761pt} % 686 (factor 2.2) + 33 head + 42 head \the\footskip
%\setlength{\marginparwidth}{7em}%
%\setlength{\marginparsep}{2em}%

% ********************************************************************
% Using different fonts
% ********************************************************************
%\usepackage[oldstylenums]{kpfonts} % oldstyle notextcomp
%\usepackage[osf]{libertine}
%\usepackage[light,condensed,math]{iwona}
%\renewcommand{\sfdefault}{iwona}
%\usepackage{lmodern} % <-- no osf support :-(
%\usepackage{cfr-lm} % 
%\usepackage[urw-garamond]{mathdesign} <-- no osf support :-(
%\usepackage[default,osfigures]{opensans} % scale=0.95 
%\usepackage[sfdefault]{FiraSans}
% ****************************************************************************************************

% ****************************************************************************************************  
% If you like the classicthesis, then I would appreciate a postcard. 
% My address can be found in the file ClassicThesis.pdf. A collection 
% of the postcards I received so far is available online at 
% http://postcards.miede.de
% ****************************************************************************************************


% ****************************************************************************************************
% 0. Set the encoding of your files. UTF-8 is the only sensible encoding nowadays. If you can't read
% äöüßáéçèê∂åëæƒÏ€ then change the encoding setting in your editor, not the line below. If your editor
% does not support utf8 use another editor!
% ****************************************************************************************************
\PassOptionsToPackage{utf8}{inputenc}
	\usepackage{inputenc}

% ****************************************************************************************************
% 1. Configure classicthesis for your needs here, e.g., remove "drafting" below 
% in order to deactivate the time-stamp on the pages
% ****************************************************************************************************
\PassOptionsToPackage{eulerchapternumbers,listings,drafting,%
					 pdfspacing,%floatperchapter,%linedheaders,%
					 subfig,beramono,eulermath,parts}{classicthesis}                                        
% ********************************************************************
% Available options for classicthesis.sty 
% (see ClassicThesis.pdf for more information):
% drafting
% parts nochapters linedheaders
% eulerchapternumbers beramono eulermath pdfspacing minionprospacing
% tocaligned dottedtoc manychapters
% listings floatperchapter subfig
% ********************************************************************


% ****************************************************************************************************
% 2. Personal data and user ad-hoc commands
% ****************************************************************************************************
\newcommand{\myTitle}{Planets orbiting evolving binary stars\xspace}
\newcommand{\mySubtitle}{Subtitle\xspace}
%\newcommand{\myDegree}{Doktor-Ingenieur (Dr.-Ing.)\xspace}
\newcommand{\myName}{Fran Bartolić\xspace}
\newcommand{\myProf}{Put name here\xspace}
\newcommand{\myOtherProf}{Put name here\xspace}
\newcommand{\mySupervisor}{Put name here\xspace}
\newcommand{\myFaculty}{Put data here\xspace}
\newcommand{\myDepartment}{Put data here\xspace}
\newcommand{\myUni}{Department of Physics, University of Rijeka\xspace}
\newcommand{\myLocation}{Rijeka, Croatia\xspace}
\newcommand{\myTime}{July 2017\xspace}
%\newcommand{\myVersion}{version 4.2\xspace}

% ********************************************************************
% Setup, finetuning, and useful commands
% ********************************************************************
\newcounter{dummy} % necessary for correct hyperlinks (to index, bib, etc.)
\newlength{\abcd} % for ab..z string length calculation
\providecommand{\mLyX}{L\kern-.1667em\lower.25em\hbox{Y}\kern-.125emX\@}
\newcommand{\ie}{i.\,e.}
\newcommand{\Ie}{I.\,e.}
\newcommand{\eg}{e.\,g.}
\newcommand{\Eg}{E.\,g.} 
% ****************************************************************************************************


% ****************************************************************************************************
% 3. Loading some handy packages
% ****************************************************************************************************
% ******************************************************************** 
% Packages with options that might require adjustments
% ******************************************************************** 
%\PassOptionsToPackage{ngerman,american}{babel}   % change this to your language(s)
% Spanish languages need extra options in order to work with this template
%\PassOptionsToPackage{spanish,es-lcroman}{babel}
	\usepackage{babel}                  

\usepackage{csquotes}
\PassOptionsToPackage{%
    %backend=biber, %instead of bibtex
	backend=bibtex8,bibencoding=ascii,%
	language=auto,%
	style=numeric-comp,%
    %style=authoryear-comp, % Author 1999, 2010
    %bibstyle=authoryear,dashed=false, % dashed: substitute rep. author with ---
    sorting=nyt, % name, year, title
    maxbibnames=10, % default: 3, et al.
    %backref=true,%
    natbib=true % natbib compatibility mode (\citep and \citet still work)
}{biblatex}
    \usepackage{biblatex}

\PassOptionsToPackage{fleqn}{amsmath}       % math environments and more by the AMS 
    \usepackage{amsmath}

\usepackage{siunitx} % for physical units
\newcommand{\vect}[1]{\boldsymbol{\mathbf{#1}}} % shorthand for bold vectors
% ******************************************************************** 
% General useful packages
% ******************************************************************** 
\PassOptionsToPackage{T1}{fontenc} % T2A for cyrillics
    \usepackage{fontenc}     
\usepackage{textcomp} % fix warning with missing font shapes
\usepackage{scrhack} % fix warnings when using KOMA with listings package          
\usepackage{xspace} % to get the spacing after macros right  
\usepackage{mparhack} % get marginpar right
%\usepackage[latest]{latexrelease} % will be used once available in more distributions (ISSUE #107)
\PassOptionsToPackage{printonlyused,smaller}{acronym} 
    \usepackage{acronym} % nice macros for handling all acronyms in the thesis
    %\renewcommand{\bflabel}[1]{{#1}\hfill} % fix the list of acronyms --> no longer working
    %\renewcommand*{\acsfont}[1]{\textsc{#1}} 
    \renewcommand*{\aclabelfont}[1]{\acsfont{#1}}
% ****************************************************************************************************


% ****************************************************************************************************
% 4. Setup floats: tables, (sub)figures, and captions
% ****************************************************************************************************
%\usepackage{tabularx} % better tables
%    \setlength{\extrarowheight}{3pt} % increase table row height
\usepackage{booktabs} % better tables
\renewcommand{\arraystretch}{1.3} (or 1.3)
\newcommand{\tableheadline}[1]{\multicolumn{1}{c}{\spacedlowsmallcaps{#1}}}
\newcommand{\myfloatalign}{\centering} % to be used with each float for alignment
\usepackage{caption}
% Thanks to cgnieder and Claus Lahiri
% http://tex.stackexchange.com/questions/69349/spacedlowsmallcaps-in-caption-label
% [REMOVED DUE TO OTHER PROBLEMS, SEE ISSUE #82]    
%\DeclareCaptionLabelFormat{smallcaps}{\bothIfFirst{#1}{~}\MakeTextLowercase{\textsc{#2}}}
%\captionsetup{font=small,labelformat=smallcaps} % format=hang,
\captionsetup{font=small} % format=hang,
\usepackage{subfig}  
% ****************************************************************************************************


% ****************************************************************************************************
% 5. Setup code listings
% ****************************************************************************************************
\usepackage{listings} 
%\lstset{emph={trueIndex,root},emphstyle=\color{BlueViolet}}%\underbar} % for special keywords
\lstset{language=[LaTeX]Tex,%C++,
    morekeywords={PassOptionsToPackage,selectlanguage},
    keywordstyle=\color{RoyalBlue},%\bfseries,
    basicstyle=\small\ttfamily,
    %identifierstyle=\color{NavyBlue},
    commentstyle=\color{Green}\ttfamily,
    stringstyle=\rmfamily,
    numbers=none,%left,%
    numberstyle=\scriptsize,%\tiny
    stepnumber=5,
    numbersep=8pt,
    showstringspaces=false,
    breaklines=true,
    %frameround=ftff,
    %frame=single,
    belowcaptionskip=.75\baselineskip
    %frame=L
} 
% ****************************************************************************************************             


% ****************************************************************************************************
% 6. PDFLaTeX, hyperreferences and citation backreferences
% ****************************************************************************************************
% ********************************************************************
% Using PDFLaTeX
% ********************************************************************
\PassOptionsToPackage{pdftex,hyperfootnotes=false,pdfpagelabels}{hyperref}
    \usepackage{hyperref}  % backref linktocpage pagebackref
\pdfcompresslevel=9
\pdfadjustspacing=1 
\PassOptionsToPackage{pdftex}{graphicx}
    \usepackage{graphicx} 
 

% ********************************************************************
% Hyperreferences
% ********************************************************************
\hypersetup{%
    %draft, % = no hyperlinking at all (useful in b/w printouts)
    colorlinks=true, linktocpage=true, pdfstartpage=3, pdfstartview=FitV,%
    % uncomment the following line if you want to have black links (e.g., for printing)
    %colorlinks=false, linktocpage=false, pdfstartpage=3, pdfstartview=FitV, pdfborder={0 0 0},%
    breaklinks=true, pdfpagemode=UseNone, pageanchor=true, pdfpagemode=UseOutlines,%
    plainpages=false, bookmarksnumbered, bookmarksopen=true, bookmarksopenlevel=1,%
    hypertexnames=true, pdfhighlight=/O,%nesting=true,%frenchlinks,%
    urlcolor=webbrown, linkcolor=RoyalBlue, citecolor=webgreen, %pagecolor=RoyalBlue,%
    %urlcolor=Black, linkcolor=Black, citecolor=Black, %pagecolor=Black,%
    pdftitle={\myTitle},%
    pdfauthor={\textcopyright\ \myName, \myUni, \myFaculty},%
    pdfsubject={},%
    pdfkeywords={},%
    pdfcreator={pdfLaTeX},%
    pdfproducer={LaTeX with hyperref and classicthesis}%
}   
\usepackage{cleveref}
% ********************************************************************
% Setup autoreferences
% ********************************************************************
% There are some issues regarding autorefnames
% http://www.ureader.de/msg/136221647.aspx
% http://www.tex.ac.uk/cgi-bin/texfaq2html?label=latexwords
% you have to redefine the makros for the 
% language you use, e.g., american, ngerman
% (as chosen when loading babel/AtBeginDocument)
% ********************************************************************
\makeatletter
\@ifpackageloaded{babel}%
    {%
       \addto\extrasamerican{%
			\renewcommand*{\figureautorefname}{Figure}%
			\renewcommand*{\tableautorefname}{Table}%
			\renewcommand*{\partautorefname}{Part}%
			\renewcommand*{\chapterautorefname}{Chapter}%
			\renewcommand*{\sectionautorefname}{Section}%
			\renewcommand*{\subsectionautorefname}{Section}%
			\renewcommand*{\subsubsectionautorefname}{Section}%     
                }%
           % Fix to getting autorefs for subfigures right (thanks to Belinda Vogt for changing the definition)
            \providecommand{\subfigureautorefname}{\figureautorefname}%             
    }{\relax}
\makeatother


% ****************************************************************************************************
% 7. Last calls before the bar closes
% ****************************************************************************************************
% ********************************************************************
% Development Stuff
% ********************************************************************
\listfiles
%\PassOptionsToPackage{l2tabu,orthodox,abort}{nag}
%   \usepackage{nag}
%\PassOptionsToPackage{warning, all}{onlyamsmath}
%   \usepackage{onlyamsmath}
% ********************************************************************
% Last, but not least...
% ********************************************************************
\usepackage{classicthesis} 
% ****************************************************************************************************


% ****************************************************************************************************
% 8. Further adjustments (experimental)
% ****************************************************************************************************
% ********************************************************************
% Changing the text area
% ********************************************************************
%\linespread{1.05} % a bit more for Palatino
%\areaset[current]{312pt}{761pt} % 686 (factor 2.2) + 33 head + 42 head \the\footskip
%\setlength{\marginparwidth}{7em}%
%\setlength{\marginparsep}{2em}%

% ********************************************************************
% Using different fonts
% ********************************************************************
%\usepackage[oldstylenums]{kpfonts} % oldstyle notextcomp
%\usepackage[osf]{libertine}
%\usepackage[light,condensed,math]{iwona}
%\renewcommand{\sfdefault}{iwona}
%\usepackage{lmodern} % <-- no osf support :-(
%\usepackage{cfr-lm} % 
%\usepackage[urw-garamond]{mathdesign} <-- no osf support :-(
%\usepackage[default,osfigures]{opensans} % scale=0.95 
%\usepackage[sfdefault]{FiraSans}
% ****************************************************************************************************

% ****************************************************************************************************  
% If you like the classicthesis, then I would appreciate a postcard. 
% My address can be found in the file ClassicThesis.pdf. A collection 
% of the postcards I received so far is available online at 
% http://postcards.miede.de
% ****************************************************************************************************


% ****************************************************************************************************
% 0. Set the encoding of your files. UTF-8 is the only sensible encoding nowadays. If you can't read
% äöüßáéçèê∂åëæƒÏ€ then change the encoding setting in your editor, not the line below. If your editor
% does not support utf8 use another editor!
% ****************************************************************************************************
\PassOptionsToPackage{utf8}{inputenc}
	\usepackage{inputenc}

% ****************************************************************************************************
% 1. Configure classicthesis for your needs here, e.g., remove "drafting" below 
% in order to deactivate the time-stamp on the pages
% ****************************************************************************************************
\PassOptionsToPackage{eulerchapternumbers,listings,drafting,%
					 pdfspacing,%floatperchapter,%linedheaders,%
					 subfig,beramono,eulermath,parts}{classicthesis}                                        
% ********************************************************************
% Available options for classicthesis.sty 
% (see ClassicThesis.pdf for more information):
% drafting
% parts nochapters linedheaders
% eulerchapternumbers beramono eulermath pdfspacing minionprospacing
% tocaligned dottedtoc manychapters
% listings floatperchapter subfig
% ********************************************************************


% ****************************************************************************************************
% 2. Personal data and user ad-hoc commands
% ****************************************************************************************************
\newcommand{\myTitle}{Planets orbiting evolving binary stars\xspace}
\newcommand{\mySubtitle}{Subtitle\xspace}
%\newcommand{\myDegree}{Doktor-Ingenieur (Dr.-Ing.)\xspace}
\newcommand{\myName}{Fran Bartolić\xspace}
\newcommand{\myProf}{Put name here\xspace}
\newcommand{\myOtherProf}{Put name here\xspace}
\newcommand{\mySupervisor}{Put name here\xspace}
\newcommand{\myFaculty}{Put data here\xspace}
\newcommand{\myDepartment}{Put data here\xspace}
\newcommand{\myUni}{Department of Physics, University of Rijeka\xspace}
\newcommand{\myLocation}{Rijeka, Croatia\xspace}
\newcommand{\myTime}{July 2017\xspace}
%\newcommand{\myVersion}{version 4.2\xspace}

% ********************************************************************
% Setup, finetuning, and useful commands
% ********************************************************************
\newcounter{dummy} % necessary for correct hyperlinks (to index, bib, etc.)
\newlength{\abcd} % for ab..z string length calculation
\providecommand{\mLyX}{L\kern-.1667em\lower.25em\hbox{Y}\kern-.125emX\@}
\newcommand{\ie}{i.\,e.}
\newcommand{\Ie}{I.\,e.}
\newcommand{\eg}{e.\,g.}
\newcommand{\Eg}{E.\,g.} 
% ****************************************************************************************************


% ****************************************************************************************************
% 3. Loading some handy packages
% ****************************************************************************************************
% ******************************************************************** 
% Packages with options that might require adjustments
% ******************************************************************** 
%\PassOptionsToPackage{ngerman,american}{babel}   % change this to your language(s)
% Spanish languages need extra options in order to work with this template
%\PassOptionsToPackage{spanish,es-lcroman}{babel}
	\usepackage{babel}                  

\usepackage{csquotes}
\PassOptionsToPackage{%
    %backend=biber, %instead of bibtex
	backend=bibtex8,bibencoding=ascii,%
	language=auto,%
	style=numeric-comp,%
    %style=authoryear-comp, % Author 1999, 2010
    %bibstyle=authoryear,dashed=false, % dashed: substitute rep. author with ---
    sorting=nyt, % name, year, title
    maxbibnames=10, % default: 3, et al.
    %backref=true,%
    natbib=true % natbib compatibility mode (\citep and \citet still work)
}{biblatex}
    \usepackage{biblatex}

\PassOptionsToPackage{fleqn}{amsmath}       % math environments and more by the AMS 
    \usepackage{amsmath}

\usepackage{siunitx} % for physical units
\newcommand{\vect}[1]{\boldsymbol{\mathbf{#1}}} % shorthand for bold vectors
% ******************************************************************** 
% General useful packages
% ******************************************************************** 
\PassOptionsToPackage{T1}{fontenc} % T2A for cyrillics
    \usepackage{fontenc}     
\usepackage{textcomp} % fix warning with missing font shapes
\usepackage{scrhack} % fix warnings when using KOMA with listings package          
\usepackage{xspace} % to get the spacing after macros right  
\usepackage{mparhack} % get marginpar right
%\usepackage[latest]{latexrelease} % will be used once available in more distributions (ISSUE #107)
\PassOptionsToPackage{printonlyused,smaller}{acronym} 
    \usepackage{acronym} % nice macros for handling all acronyms in the thesis
    %\renewcommand{\bflabel}[1]{{#1}\hfill} % fix the list of acronyms --> no longer working
    %\renewcommand*{\acsfont}[1]{\textsc{#1}} 
    \renewcommand*{\aclabelfont}[1]{\acsfont{#1}}
% ****************************************************************************************************


% ****************************************************************************************************
% 4. Setup floats: tables, (sub)figures, and captions
% ****************************************************************************************************
%\usepackage{tabularx} % better tables
%    \setlength{\extrarowheight}{3pt} % increase table row height
\usepackage{booktabs} % better tables
\renewcommand{\arraystretch}{1.3} (or 1.3)
\newcommand{\tableheadline}[1]{\multicolumn{1}{c}{\spacedlowsmallcaps{#1}}}
\newcommand{\myfloatalign}{\centering} % to be used with each float for alignment
\usepackage{caption}
% Thanks to cgnieder and Claus Lahiri
% http://tex.stackexchange.com/questions/69349/spacedlowsmallcaps-in-caption-label
% [REMOVED DUE TO OTHER PROBLEMS, SEE ISSUE #82]    
%\DeclareCaptionLabelFormat{smallcaps}{\bothIfFirst{#1}{~}\MakeTextLowercase{\textsc{#2}}}
%\captionsetup{font=small,labelformat=smallcaps} % format=hang,
\captionsetup{font=small} % format=hang,
\usepackage{subfig}  
% ****************************************************************************************************


% ****************************************************************************************************
% 5. Setup code listings
% ****************************************************************************************************
\usepackage{listings} 
%\lstset{emph={trueIndex,root},emphstyle=\color{BlueViolet}}%\underbar} % for special keywords
\lstset{language=[LaTeX]Tex,%C++,
    morekeywords={PassOptionsToPackage,selectlanguage},
    keywordstyle=\color{RoyalBlue},%\bfseries,
    basicstyle=\small\ttfamily,
    %identifierstyle=\color{NavyBlue},
    commentstyle=\color{Green}\ttfamily,
    stringstyle=\rmfamily,
    numbers=none,%left,%
    numberstyle=\scriptsize,%\tiny
    stepnumber=5,
    numbersep=8pt,
    showstringspaces=false,
    breaklines=true,
    %frameround=ftff,
    %frame=single,
    belowcaptionskip=.75\baselineskip
    %frame=L
} 
% ****************************************************************************************************             


% ****************************************************************************************************
% 6. PDFLaTeX, hyperreferences and citation backreferences
% ****************************************************************************************************
% ********************************************************************
% Using PDFLaTeX
% ********************************************************************
\PassOptionsToPackage{pdftex,hyperfootnotes=false,pdfpagelabels}{hyperref}
    \usepackage{hyperref}  % backref linktocpage pagebackref
\pdfcompresslevel=9
\pdfadjustspacing=1 
\PassOptionsToPackage{pdftex}{graphicx}
    \usepackage{graphicx} 
 

% ********************************************************************
% Hyperreferences
% ********************************************************************
\hypersetup{%
    %draft, % = no hyperlinking at all (useful in b/w printouts)
    colorlinks=true, linktocpage=true, pdfstartpage=3, pdfstartview=FitV,%
    % uncomment the following line if you want to have black links (e.g., for printing)
    %colorlinks=false, linktocpage=false, pdfstartpage=3, pdfstartview=FitV, pdfborder={0 0 0},%
    breaklinks=true, pdfpagemode=UseNone, pageanchor=true, pdfpagemode=UseOutlines,%
    plainpages=false, bookmarksnumbered, bookmarksopen=true, bookmarksopenlevel=1,%
    hypertexnames=true, pdfhighlight=/O,%nesting=true,%frenchlinks,%
    urlcolor=webbrown, linkcolor=RoyalBlue, citecolor=webgreen, %pagecolor=RoyalBlue,%
    %urlcolor=Black, linkcolor=Black, citecolor=Black, %pagecolor=Black,%
    pdftitle={\myTitle},%
    pdfauthor={\textcopyright\ \myName, \myUni, \myFaculty},%
    pdfsubject={},%
    pdfkeywords={},%
    pdfcreator={pdfLaTeX},%
    pdfproducer={LaTeX with hyperref and classicthesis}%
}   
\usepackage{cleveref}
% ********************************************************************
% Setup autoreferences
% ********************************************************************
% There are some issues regarding autorefnames
% http://www.ureader.de/msg/136221647.aspx
% http://www.tex.ac.uk/cgi-bin/texfaq2html?label=latexwords
% you have to redefine the makros for the 
% language you use, e.g., american, ngerman
% (as chosen when loading babel/AtBeginDocument)
% ********************************************************************
\makeatletter
\@ifpackageloaded{babel}%
    {%
       \addto\extrasamerican{%
			\renewcommand*{\figureautorefname}{Figure}%
			\renewcommand*{\tableautorefname}{Table}%
			\renewcommand*{\partautorefname}{Part}%
			\renewcommand*{\chapterautorefname}{Chapter}%
			\renewcommand*{\sectionautorefname}{Section}%
			\renewcommand*{\subsectionautorefname}{Section}%
			\renewcommand*{\subsubsectionautorefname}{Section}%     
                }%
           % Fix to getting autorefs for subfigures right (thanks to Belinda Vogt for changing the definition)
            \providecommand{\subfigureautorefname}{\figureautorefname}%             
    }{\relax}
\makeatother


% ****************************************************************************************************
% 7. Last calls before the bar closes
% ****************************************************************************************************
% ********************************************************************
% Development Stuff
% ********************************************************************
\listfiles
%\PassOptionsToPackage{l2tabu,orthodox,abort}{nag}
%   \usepackage{nag}
%\PassOptionsToPackage{warning, all}{onlyamsmath}
%   \usepackage{onlyamsmath}
% ********************************************************************
% Last, but not least...
% ********************************************************************
\usepackage{classicthesis} 
% ****************************************************************************************************


% ****************************************************************************************************
% 8. Further adjustments (experimental)
% ****************************************************************************************************
% ********************************************************************
% Changing the text area
% ********************************************************************
%\linespread{1.05} % a bit more for Palatino
%\areaset[current]{312pt}{761pt} % 686 (factor 2.2) + 33 head + 42 head \the\footskip
%\setlength{\marginparwidth}{7em}%
%\setlength{\marginparsep}{2em}%

% ********************************************************************
% Using different fonts
% ********************************************************************
%\usepackage[oldstylenums]{kpfonts} % oldstyle notextcomp
%\usepackage[osf]{libertine}
%\usepackage[light,condensed,math]{iwona}
%\renewcommand{\sfdefault}{iwona}
%\usepackage{lmodern} % <-- no osf support :-(
%\usepackage{cfr-lm} % 
%\usepackage[urw-garamond]{mathdesign} <-- no osf support :-(
%\usepackage[default,osfigures]{opensans} % scale=0.95 
%\usepackage[sfdefault]{FiraSans}
% ****************************************************************************************************


%********************************************************************
% Bibliographies
%*******************************************************
\addbibresource{library}
\addbibresource{books}
\begin{document}
\frenchspacing
\raggedbottom

%\renewcommand*{\bibname}{new name}
%\setbibpreamble{}
\pagenumbering{roman}
\pagestyle{plain}

\chapter{N-body simulations of circumbinary systems up to the Common Envelope phase}
\label{ch:NBODY_simulations}
%************************************************
While analytic models can give important insight into the physics 
of divergent resonance passage in a circumbinary planetary system
they cannot compare to the full solution of the equations of motion.
Here I describe a different approach to studying evolving
circumbinary systems, by means of direct
N-body simulations coupled with simulations of the stellar binary. 
These simulations provide a complete picture of the dynamical evolution
of the system. To realistically simulate the dynamics of a stellar binary
with a circumbinary planet as the stellar binary evolves, we need to know
how to stars evolve at each time step and how they interact with the outer
planet(s). To solve this computationally we need to couple in some way
a stellar evolution code with an N-body solver. 

\section{The N-body problem}
\label{sec:The N-body problem}
The \emph{N-body problem} is a problem of $N$ gravitationally interacting
masses. The equations of motion are
\begin{equation}
    m_i\ddot{\vect{r}}_i=\sum_{j=1,j\neq 1}^NGm_im_j \frac{\vect{r}_{ij}}
    {\lvert \vect{r}_{ij}\rvert^3} 
    \label{eq:nbody_equations}
\end{equation}
where $m_i$ is the mass of the $i$-th particle, $\vect{r}_i$ is its 
position vector, $\vect{r}_{ij}=\vect{r}_j-\vect{r}_i$ and $i=1\dots N$.
Once the initial positions $\vect{r}_{i,0}$ and velocities $\dot{\vect{r}}_{i,0}$
are specified, there exists a unique solution which can only be obtained
numerically for $N>2$. The N-body system of equations is extremely sensitive
to the initial conditions and special care must be taken to ensure numerical
accuracy of the soltions. Since \cref{eq:nbody_equations} involve the force
calculation between each pair of $N$ particles, the computational complexity
of the problem scales as $\mathcal{O}(N^2)$.

There are many different ways of solving \cref{eq:nbody_equations} numerically,
and the choice of method depends primarily on the problem at hand. For systems
with large $N$
approximation schemes are often used which manage to reduce the complexity
to order $\mathcal{O}(N\ln N)$. In celestial mechanics $N$ is generally on
the order of a few and the integration times are very long (sometimes 
billions of years), high precision is thus required and the complexity is 
$\mathcal{O}(N^2)$. Another difficulty is the fact that \cref{eq:nbody_equations}
has a singularity when two particles come very close to each other which
can lead to arbitrarily high particle velocities if not properly handled.
For a system of $N$ gravitationally interacting bodies the total energy 
and angular momentum are always conserved, however, this need not be true 
for a numerical solution of the equations of motion. Often, the 'quality'
of an integrator\footnote{An \emph{integrator} is a name often given to
any numerical scheme which solves a system of differential equations.}
is judged by how good it conserves the total energy.

A class of integrators called \emph{symplectic integrators} 
\citep{vogelaere,ruth,feng} is often used
in celestial mechanics. Instead of solving \cref{eq:nbody_equations}, these
integrators solve the equivalent Hamiltonian system. They preserve
invariant propertties such as phase-space density and in many cases
can have an upper bound on the energy error.
\section{REBOUND - an open source N-body integrator}
\label{sec:REBOUND - an open source N-body integrator}
Because the N-body problem occurs very often in astrophyisical applications,
there exist multiple numerical solvers for problems of various scales and 
with varying degree of accuracy. Traditionally, the code most often used for applications
in celestial mechanics is \texttt{MERCURY} \citep{chambers1997,chambers1999}, 
written in \texttt{Fortran} programming language.
For the problem of simulating a circumbinary system
which involves pure N-body computations and dissipative forces, I have 
opted to use a different code called \texttt{REBOUND} \citep{Rein2012}.
REBOUND is an open-source code freely available on \texttt{GitHub}\footnote{
    \url{https://github.com/hannorein/rebound}}, it is written in \texttt{C99}
and also has a \texttt{Python} interface, it is under active development.
The reasons for choosing \texttt{REBOUND} over \texttt{MERCURY} are several,
most important being the ease of use, the availability of 
integrators accurate to machine precision  and the fact that 
\texttt{REBOUND} uses Jacobi coordinates
by default (which matches well with the analysis in \cref{ch:analytical_model})
while \texttt{MERCURY} uses heliocentric coordinates which are more suitable to
Solar System type of problems.

\texttt{REBOUND} comes with many integrators, the most important of which
are the symplectic integrator \texttt{WHFAST} \citep{Rein2015} and an
adaptive high-order integrator called \texttt{IAS15} \citep{Rein2014}. For
applications which involve velocity-dependent non-conservative forces
such as tidal decay of binary stars, \texttt{IAS15} is a better choice
for such applications because symplectic integrators are by construction
designed for conservative forces. 

\texttt{IAS15} is designed to solve a system of differential equations of the form
\begin{equation}
    \ddot{\vect{y}}(t)=F(\dot{\vect{y}},\vect{y},t)
    \label{eq:diff_eq}
\end{equation}
where $\ddot{\vect{y}}$ is the acceleration and $F$ is an arbitrary force
which may depend also on the velocity. \Cref{eq:diff_eq} can be expanded
into a truncated series:
\begin{equation}
    \ddot{\vect{y}}(t)\approx \ddot{\vect{y}}_0 + \vect{a}_0t+\vect{a}_1t^2
    +\dots+\vect{b}_6t^7
\end{equation}
where $\ddot{\vect{y}}_0$ is the acceleration evaluated at $t=0$. 
It can then be rewritten as
\begin{equation}
    \ddot{\vect{y}}(h)\approx \ddot{\vect{y}}_0 + \vect{b}_0h+\vect{b}_1h^2
    +\dots+\vect{b}_6h^7
    \label{eq:diff_eq2}
\end{equation}
where $dt$ is the \emph{timestep}, $h=t/dt$ and $\vect{b}_k=\vect{a}_kdt^{k+1}$.
Finally, we rewrite it once again as
\begin{equation}
    \ddot{\vect{y}}(h)\approx \ddot{\vect{y}}_0 + \vect{g}_1h+\vect{g}_2h
    (h-h_1)+\vect{g}_3h(h-h_1)(h-h_2)+\dots+g_8h(h-h_1)\dots(h-h_7)
\end{equation}
Where the timestep $h$ is now divided into substeps $h_1,\dots,h_7$ in the
interval $[0,1]$. The coefficients $\vect{g}_k$ can be determined from 
$\vect{b}_k$. When the expansion is written in this form, the coefficient 
$\vect{g}_k$ depends only on the force evaluations at previous substeps
$h_n$ with $n\leq k$. \Cref{eq:diff_eq2} can then be integrated to give the velocity
\begin{equation}
    \dot{\vect{y}}\approx \dot{\vect{y}}_0+hdt\left(\ddot{\vect{y}}_0+
    \frac{h}{2} \left(\vect{b}_0+ \frac{2h}{3} \left(\vect{b}_1+\dots\right)
    \right)\right)
\end{equation}
and once again for the position
\begin{equation}
    \vect{y}\approx\vect{y}_0+\cdot{\vect{y}}_0hdt+ \frac{h^2dt^2}{2} 
    \left(\ddot{\vect{y}}_0+
    \frac{h}{3} \left(\vect{b}_0+ \frac{h}{2} \left(\vect{b}_1+\dots\right)
    \right)\right)
\end{equation}
To get the new positions and velocities, we need to estimate the coefficients
$\vect{g}_k$ (and hence also $\vect{b}_k$). These are obtained using an implicit
scheme
\citep[see][for details]{Rein2014}. To obtain the best possible accuracy for
the position and the velocity, the spacing of substeps in \texttt{IAS15} 
\cite{Rein2014} chose Gauss-Radau spacing\footnote{Gauss-Radau spacing is related to 
the Gaussian quadrature which is a way of approximating a solution to an 
integral on a definite interval} with $h_n=0, 0.0562, 0.1802, 0.3526, 
0.5471, 0.7342, 0.8853 and 0.9775$. In a Gauss-Radau integration scheme
where a function $F(t)$ is integrated on the domain $[0,dt]$ with $m$ quadrature
points and the absolute error term is of order $\mathcal{O}(dt^{2m})$
\cite{Rein2014}. 
For $m=8$ the error term is $\mathcal{O}(dt^{16})$ which makes this
a 15-th order scheme, hence the name \texttt{IAS15}. The fact that this
integrator is 15th order means that by reducing the timestep by a factor of
$\alpha$ we reduce the error by a factor of $\alpha^{16}$! Thus, to obtain
an accurate solution a very small timestep is often not necessary, it is
usually sufficient to take a fraction of the smallest relevant dynamical
timescale in the problem.

The \emph{total error} in for example energy or position 
depends not only on the error associated
with the truncation of an infinite series in a numerical scheme but also
on error $E_\text{floor}$ associated with the finite precision of 
floating-point numbers 
(\texttt{IAS15} uses \texttt{double} precision numbers) also known as
roundoff error, a random error 
$E_\text{random}$ coming from an addition of two floating-point numbers 
and a bias error $E_\text{bias}$ associated
with the bias of floating-point operations. That is, the total error $E$ is
\begin{equation}
    E=E_\text{floor}+E_\text{rand}+E_\text{bias}+E_\text{scheme}
\end{equation}
The total error is then dominated by the largest term. 

\texttt{IAS} uses an adaptive timestepping scheme in which a new time
step is chosen after previous timestep, since it is very high-order, the 
timestep can be chosen such that the error associated with the scheme
$E_\text{scheme}$ remains below $10^{-16}$, which is less than the precision
of \texttt{double} floating-point numbers. We say that the integrator is
accurate to machine precision. Therefore, it is not possible to achieve
better precision with a different integrator unless the floating-point 
precision is extended (to say 128 bit numbers). 
The total error then depends on the other errors. $E_\text{floor}$ is constant,
and $E_\text{bias}$ in general might grow linearly with time because
some floating-point operations might be biased, for example, in a series
of repeated additions the error might be prefferentialy positive instead of
either positive or negative with equal probability. To avoid these types of
errors all mathematical operations in \texttt{IAS15} use only the operators
$+,-,\times,/$ which are guaranteed to give the same result\footnote{This is 
in general not true for other operations such as $\sqrt{x}$ which may give 
slightly different results depending on the system architecture.}in \texttt{C99}
independent of the hardware on which the code is run or the compiler differences.
Even if such biases are removed and the error in each individual calculation is
completely random, the error (designated $E_\text{rand}$) grows with time. 
\cite{brouwer} showed that  quantities such as the energy  or position error
grow as $\propto t^{1/2}$ and the error in angles grows as $\propto t^{3/2}$,
this is known as \emph{Brouwer's law}.

Extensive test of the \texttt{IAS15} integrator presented in \cite{Rein2014} 
show that it follows Brouwer's law, that is, the error is dominated by the 
$E_\text{random}$ term. The closest comparable integrator, \texttt{RADAU}
in \texttt{MERCURY} generally has a two orders of magitude greater error. 

\section{binary\_c - a binary stellar evolution code}
\label{sec:binary_c - a binary stellar evolution code}
If we were interested in the dynamics of circumbinary system on the main 
sequence an N-body integrator would be sufficient because purely gravitational
forces dominate. To investigate the dynamics beyond the main sequence we 
need some prescription the evolution of the stellar binary. 

I opted to use the \texttt{binary\_c}
\footnote{\url{http://www.ast.cam.ac.uk/~rgi/binary_c.html}}
stellar evolution code described in
\cite{izzard2004,izzard2006,izzard2009}, which is based on a binary
stellar evolution code \texttt{BSE} \citep{hurley2002}. Both \texttt{BSE}
and \texttt{binary\_c} do not solve the equations of stellar structure directly
(which is very computationally expensive) but rather uses analytical formulae
fitted to the outputs of detailed models which do solve the stellar structure
equations. Since analytic expressions are computationally 'cheap' to evaluate,
the codes are very fast. \texttt{binary\_c}
is unfortunately not freely available for download, however it is available
upon request from the author. The code is written in \texttt{C} and it
includes an \texttt{API} interface for calling functions which 
evolve the binary system for specified initial conditions and output the
physical variables for each time step. \Cref{tab:binary_c_params} shows
the input parameters which need to be specified in order to run the code.
\begin{table}[h!]
\centering
\begin{tabular}{lc}
\toprule
    Parameter & meaning\\
\midrule
    \texttt{M1} & mass of primary star\\
    \texttt{M2} & mass of secondary star\\
    \texttt{PER} & initial orbital period of the binary in days\\
    \texttt{ECC} & initial eccentricity\\
    \texttt{Z} & metallicity\\
    \texttt{EVOLTIME} & total evolution time\\
\bottomrule
\end{tabular}
    \caption{Initial parameters for the stellar evolution code \texttt{binary\_c}.}
\label{table:binary_c_params}
\end{table}
For all the other parameters we keep the default values specified in the code, for 
the metallicity we choose a fixed value of solar-like metallicity of $0.02$ and 
evolve all systems for a duration of a Hubble time (13.7 Gyr).

In order to later on couple the \texttt{binary\_c} code with \texttt{REBOUND},
and easily plot the output from the code, I have written a \texttt{Python}
interface to the code. The interface is built using the \texttt{Python} 
library \texttt{ctypes} which enables one to access \texttt{C} functions
from within \texttt{Python} if the \texttt{C} code is compiled as a \emph{
    shared library}. I use \texttt{ctypes} to access \texttt{binary\_c API}
functions from within \texttt{Python} and grab the output directly into memory
as a \texttt{Python} object.
\begin{figure}[htb]
\centering
\includegraphics[width=0.7\linewidth]{gfx/stellar_evo_ex.pdf}
\caption{An example of the stellar evolution output data from the \texttt{binary\_c}
    code for $m_1=1.2\,M_\odot$, $m_2=0.5\,M_\odot$, $e=0.3$, $P=100$ days. Left
    panel shows the time evolution of the stellar radii and the right panel shows
    the evolution of orbital parameters.}
\label{fig:stellar_evo_ex}
\end{figure}

\Cref{fig:stellar_evo_ex} shows an example output of the \texttt{binary\_c} code
for an eccentric short-period binary ($e=0.3$, $P=100$ days), consisting of 
main-sequence star of mass  
$m_1=1.2\,M_\odot$, and a red-dwarf of mass $m_2=0.5\,M_\odot$. 
The final outcome of the stellar evolution in this case is a binary 
consisting of a white dwarf and a red dwarf.
The top panel shows the evolution of the radius in time. It is apparent that 
the higher mass star ascends the red giant branch first, while the lower mass
star stays at near constant radius. The stellar radius is an important parameter
because tidal forces which drive the orbital evolution depend very strongly on 
radius. It is also important because for close binaries with a period shorter
than about 1000 days, the stellar radius of the primary at the tip of the 
red giant branch (the sharp peak in the figure) reaches beyond the Roche
Lobe radius of the primary which triggers the formation of a common envelope. 

The middle panel shows the masses of the two stars as a function of time.
The mass loss is for the purposes of dynamical modeling negligible up to 
the CE phase (less than 1\% for the primary star and near zero for the 
secondary). During the CE phase a large percentage of the total mass in 
the system is suddenly ejected. And then again nothing interesting
happens because the secondary in this case is a low-mass star.

The orbital evolution of the binary in period-eccentricity space is shown 
in the bottom panel of \cref{fig:stellar_evo_ex}. 
Initially the eccentricity is $e=0.3$ and the period is 100 days, 
very quickly however the orbit is circularized via tidal 
interactions and it continues to shrink at near zero eccentricity until
a common-envelope event is triggered. Because resonance widths tend to 
zero as the binary eccentricity declines, the influence of resonance
passage on circumbinary planets is most significant if the passage happens
at non-zero binary eccentricity. \Cref{fig:stellar_evo_kepler} shows the 
evolutionary tracks in period-eccentricity space of stars in observed 
 main sequence circumbinary systems (see \cref{tab:kepler_planets}). 
 Shown are only the tracks
from those stars which experience significant orbital evolution within a 
Hubble time. Wee see that only a single circumbinary planet, PH-1 A (ab) b,
crosses a resonance at an appreciable eccentricity of the stellar binary,
namely the 7:1 MMR. All the other planets either cross the resonances
after the binary's orbit has circularized or they orbit outside of even
the $9:1$ MMR location. Thus, only a small subset of the initial parameter
space results in outcomes which favor resonance crossing scenarios likely
to cause significant eccentricity excitation. Nevertheless, the sample of 
observed circumbinary systems in \cref{tab:kepler_planets} is very small 
and there might stil be a significant population of circumbinary planets 
located just inside the the $6:1$ and $7:1$ MMR locations.
\begin{figure}[htb]
\centering
\includegraphics[width=\linewidth]{gfx/stellar_evo_kepler.pdf}
\caption{Evolutionary tracks of circumbinary planet hosting main-sequence
    binaries in period-eccentricity space. The vertical lines show the location
     of high-order mean motion commensurabilities up to the $9:1$ commensurability
     . The title above each panel
     shows the name of the stellar binary and the planet.}
\label{fig:stellar_evo_kepler}
\end{figure}

Having covered binary evolution up to the common envelope phase, the question
remains what happens during and after the common envelope.
The common envelope is a very short lived and poorly understood phase of 
binary stellar evolution (for a recent review see \cite{Ivanova2013}). 
Depending on the model it lasts on the order of a few
tenths to a few thousands of years, a blink of an eye when compared to 
typical stellar evolution timescales. 
As mentioned in \cref{sec:stellar_evolution}, it is modeled with 
with a simple $\alpha$ parameter which tells us what percentage of the orbital 
energy went into ejection of the common envelope. By default 
\texttt{binary\_c} uses a value of $\alpha_\text{CE}=3.0$. Most stellar evolution codes,
including \texttt{binary\_c}, do not resolve the CE phase but just proceed to
calculate the evolution after ejection, accounting for the mass loss and the 
possiblity of a merger during the CE. To get an idea of what the most common outcomes
of common envelope evolution are, in \cref{fig:evolution_outcomes} we plot a density 
map of \texttt{binary\_c} outcomes classified into four categories:
\begin{enumerate}
    \item None - CE did not occur
    \item CE - CE did occur and the final system is a binary
    \item CE \& merger - CE occured and the two stars merged into a single star
    \item SN - CE occured and the two stars underwent a double-degenerate supernova
\end{enumerate}
Left panel of \cref{fig:evolution_outcome} shows a stellar evolution 
outcomes for very short period binaries with
$P=10$ days, similar to the binaries observed in circumbinary systems.  
It appears that all main-sequence
binary stars similar to the observed circumbinary systems end up as a single star, a 
product of a stellar merger event during the CE phase. 
For the somewhat longer period binaries (right panel), the part of parameter space
which does not result in a merger is slightly larger and also possible for low 
mass binaries. In general, these results should be taken with a grain of salt 
because the physics behind the CE phase is so uncertain. We can however conclude
that all of the circumbinary planet hosting binaries do undergo a common envelope
phase, the question is just wheater they end up as a single star or a very tight 
binary.
\begin{figure}[t!]
\centering
\begin{subfigure}{0.52\textwidth}
  \centering
  \includegraphics[width=1\linewidth]{gfx/outcomes_10.pdf}
%  \caption{A subfigure}
%  \label{fig:sub1}
\end{subfigure}%
\begin{subfigure}{0.52\textwidth}
  \centering
  \includegraphics[width=1\linewidth]{gfx/outcomes_100.pdf}
%  \caption{A subfigure}
%  \label{fig:sub2}
\end{subfigure}
    \caption{Final outcomes of stellar evolution generated by the 
    \texttt{binary\_c} code as a function of the the two masses. Left panel: 
    binary with an initial period of $P=100$ days and initial eccentricity. 
    Right panel: a 
    binary with an initial period of $P=10$ days. Both panels show systems 
    with an initial eccentricity of $e=0.3$. Various outcomes are denoted
    by different colors specified in the legend.}
\label{fig:evolution_outcomes}
\end{figure}

\section{Stability of circumbinary systems on the main sequence}
\label{sec:Stability of observed circumbinary planets on the main sequence}
Before moving  to the simulations of circumbinary planets around evolving
binaries, we investigate the stability during the main sequence. A reasonable
strategy is to use \texttt{REBOUND} and integrate systems with various initial
conditions for a certain period of time, say 10 million years. There is however
a better and computationally cheaper way of assessing stability. As mentioned
in \cref{ch:theoretical_background}, phase-space trajectories with arbitrarily 
mall differences in their initial conditions are said to be 
\emph{chaotic} if they tend to diverge 
from each other exponentially.  Measuring the rate of this exponential 
divergence gives us some estimate of how 'chaotic' the trajectories are.

The system of N-body equations \cref{eq:nbody_equations} is of the form
\begin{equation}
    \dot{\vect{x}}=f(\vect{x}),\vect{x}\in\mathbb{R}^{6N}
    \label{eq:nbody_eq_general}
\end{equation}
The solution to these equations defines a trajectory $\vect{x}(t)$ in phase-space.
Consider now a trajectory $\vect{x}'$ initially close to $\vect{x}$,
$\vect{x}'(0)=\vect{x}(0)+\delta\vect{x}(0)$. To measure their relative divergence,
we cannot just study the Eucledian norm $\lVert \delta\vect{x}(t)\rVert$ as 
a function of time because in the case of bounded motion, this distance cannot
grow indefinitely. What we need is is a measure of \emph{local} exponential
divergence of nearby trajectories. Since local means in the vicinity of a point
in phase space $\vect{x}$, we can \emph{linearize} the equations
of relative motion by expanding the function $f(\delta\vect{x})$ around
zero to first order. The linearized equations of relative motion, also called
the \emph{variational equations}, then have the form
\begin{equation}
    \dot{\boldsymbol{\delta}}=A(t)\boldsymbol{\delta}
    \label{eq:variational_equations}
\end{equation}
where the vector $\boldsymbol{\delta}(t)\equiv\delta\vect{x}(t)$ measures the
Euclidian distance between two neighboring trajectories in phase-space
and $A$ is the Jacobian matrix with elements $A_{ij}=\partial f_i/\partial x_j$.
For as long as $\delta\vect{x}(t)$
stays small, the linearized equations are a good approximation of the motion. The
\emph{Maximum Lyapunov Exponent} is then defined as \citep{Hinse2010,morbidelli2002}
\begin{equation}
    \gamma=\lim_{t\rightarrow\infty} \frac{1}{t-t_0} \ln\left( \frac{\lVert 
    \boldsymbol{\delta}(t)\rVert}{\lVert \boldsymbol{\delta}(t_0)\rVert}\right)=
    \lim_{t\rightarrow\infty} \frac{1}{t-t_0}\int^t_{t_0} 
    \frac{\lVert \dot{\boldsymbol{\delta}}(s)\rVert}{\lVert 
    \boldsymbol{\delta}(s)\rVert}\mathrm{d}s
\end{equation}
 The ratio $\lVert\dot{\boldsymbol{\delta}}/ \boldsymbol{\delta}\rVert$ measures the
rate of change of the separation vector $\boldsymbol{\delta}$. The rate of change
of $\boldsymbol{\delta}$ is then $e^{\gamma t}$. If $\gamma >0$, the initial separation
grows exponentially in time and we have chaotic motion. If $\gamma=0$ the separation
does not change and we have quasi-periodic or regular motion. Finally, for $\gamma<0$
the two trajectories approach each other at an exponential rate, this only happens
for dissipative systems because it implies the existance of attractive fixed points which
act as sinks in phase space. This is not possible for conservative Hamiltonian
systems since it is a violation of Liouville's theorem. 

In practice, calculating the limit when $t\rightarrow\infty$ is 
computationally not feasible and usually one takes a 
sufficiently long time $t=t_\text{max}$. For a recent review on the calcualtion
of maximum Lyapunov exponents see \cite{skokos}.
There are various ways of more efficiently computing either the MLE or so called
chaos indicators closely related to the MLE. One such indicator is called the 
\emph{Mean Exponential Growth of Nearby Orbits} \citep{refId0,cincotta2003} or MEGNO for short. 
It is defined as
\begin{equation}
    Y(t)= \frac{2}{t} \int^t_{t_0} \frac{\lVert \dot{\boldsymbol{\delta}}(s)\rVert}{\lVert 
    \boldsymbol{\delta}(s)\rVert}s\mathrm{d}s
\end{equation}
and its time-avaraged mean value is 
\begin{equation}
    \langle Y\rangle\left( t\right)= \frac{1}{t} \int^t_{t_0} Y(s)\mathrm{d}s
\end{equation}
The difference between the MLE and the MEGNO indicator is that the integral in MEGNO
is weighted by time during the integration, thus giving more wait to the rate 
of divergence of the separation vector at later times. Numerically, the MEGNO
is computed by integrating the variational equations \cref{eq:variational_equations}
wit a random initial displacement vector $\boldsymbol{\delta}_0$
during the time interval $t-t_0$. Most other chaos indicators based on the
MLE do not have this weighting. 
Generally, $\langle Y\rangle$ can be parametrized as a linear function in time
$\langle Y\rangle (t)=\alpha t+\beta$. \cite{cincotta2003} show that if the 
motion is quasi-periodic $\alpha\approx 0$ and $\beta\approx 2$ and 
$\langle Y\rangle \rightarrow 2$ as $t\rightarrow \infty$. Indeed, it
converges to zero than the MLE \citep{cincotta2003}. In the case of a chaotic
orbit $Y$ and $\langle Y\rangle$ increase linearly with time ($\alpha\neq 0$).

The relationship between chaotic orbits (high MEGNO) and \emph{stability} of
a system is not straightforward. A system is deemed stable for a certain time 
if there are no \emph{close encounters}  between any two bodies in the system 
leading to hyperbolic orbits. Chaotic trajectories need not lead to ejections
because the chaos might be \emph{bounded}. Bounded chaos means that even
if at a certain point in phase-space we have exponentially diverging 
trajectories, the trajectories are still bounded to a certain \emph{finite} 
volume of phase-space. Thus, high MEGNO values can only be used as a proxy
for instability. We should always keep that caveat in mind when 
interpreting MEGNO maps (or any other chaos indicator). For a discussion about 
the relationship between chaos,
 resonance overlap and  stability, see \cite{Deck2013}.

Since MEGNO is calculated for a finite integration
time, there
is always a possibility that some trajectories start quasi-period and suddenly
become chaotic after a time longer than the integration time. However, if an orbit
has a large MEGNO value after a small interval of time, it 
is almost certainly chaotic for all times. MEGNO is ideal for revealing 
regions of phase-space which are chaotic due to overlapping mean-motion resonances.

The question remains what is a suitable integration time for calculating the 
MEGNO values for a given system. There is no unique answer as it depends on 
the problem. For our purposes, the answer is  - whatever time is enough 
to reveal the fine structure in the space of orbital parameters 
due to resonance overalap. If we were to calculate the MEGNO values
for say the Solar System and choose and extremely long integration time
on the order of billions of years, we would find that almost all initial
conditions result in chaotic orbit because even though the Solar System looks
stable now, it can be chaotic on a sufficiently long timescale.

The \texttt{REBOUND} package includes functions to calculate
the MEGNO value out of the box. It does by integrating the variational equations
for a specified period of time, either with the high-order \texttt{IAS15} 
integrator or the symplectic \texttt{WHFAST}. Since we are interested
in the stability of circumbinary systems on the main-sequence which are
(to a good degree of approximation) conservative, we opt for the 
\texttt{WHFAST} integrator since it is faster than \texttt{IAS15}. We chose
to calculate MEGNO as a function of two orbital parameters because we can 
then construct a density plot which is more informative than a one 
dimensional function.
\begin{figure}[htb]
\centering
\includegraphics[width=\linewidth]{gfx/megno_time_comparison.pdf}
\caption{MEGNO maps for a system with 
    $m_1=1\,M_\odot$, $m_2=1\times 10^{-3}\,M_\odot$, $m_3=m_2$,
    $P_i=10$ days, $e_i=0.1$ and $\varpi_o-\varpi_i=0$, c
    alculated on a $150\times 150$
    grid in variables $(P_o/P_i,e_o)$. White points are regular orbits
    and the dark grey points are chaotic orbits. The left panel
    shows a MEGNO map for an integration time of $1000$ years and the
    right one for an integration time of $5000$ years. The dashed white line in
    the top left corner of each panel is defined 
    by $R_p/a_i=1$ where $R_p=a_o(1-e_o)$
    is the outer periastron distance. Every outer orbit above that line
    is crossing the inner orbit and is therefore guaranteed to be 
    unstable in almost all cases.}
\label{fig:megno_example}
\end{figure}

\Cref{fig:megno_example} shows an example of MEGNO maps for a system
with $m_1=1\,M_\odot$, $m_2=1\times 10^{-3}\,M_\odot$ and $m_3=m_2$,
that is, a system with a single star and two Jupiter mass planets.
We choose $P_i=10$, $e_i=0.1$, $\varpi_o-\varpi_i=0$, and 
we vary $e_o$ and $P_o$.
Both planets have the same  value of the argument
of pericentre. We
could have just as well chosen some other parameters, these are just
the most useful ones to plot.  The left panel shows a MEGNO map for a total 
integration time of 1000 years. The $n:1$ 
resonances are easily noticable, it also appears that the region
up to the $2:1$ resonance is mostly unstable. The panel on the right
shows the same MEGNO map with a longer integration time of 5000yr.
As expected, the resonances are more easily apparent, however, the
computational demands for a longer integration are considerable (an 
hour on a 32 core machine with a paralellized code) and 
we opt to use an integration time of around 1000 yr. Both plots
show an unusual strip of regular orbits at very high eccentricities.
This is most likely due the linear approximation of the variational
equations used to calculate the MEGNO values. The variational equations
are valid only in the vicinity of the point at which we expand the
full equations of motion, presumably it fails in this particular 
region because at those high eccentricities the planets are likely
to experience close encounters which significantly perturb both 
planets and the linear approximation breaks down. Above the strip at
even higher eccentricities the orbits are again chaotic because at
that point the close encounters actually result in an ejection event
and an ejection event automatically assigns a maximum value of 
MEGNO.

After introducing the MEGNO technique on an example of a single-star 
system with two planets, we apply it to circumbinary systems.
\Cref{fig:megno_kepler} shows MEGNO maps for the observed circumbinary
systems with relatively close-in planets. Plotted on top of the 
MEGNO results are the analytically derived resonace widths (blue 'bubbles'
in the figure) from
\cite{Mardling2013} (see. \cref{ssub:Resonance_widths} for derivation), 
the red dots are the observationally derived values for the 
planet eccentricity and period. The first thing one should notice when
comparing the various panels in \cref{fig:megno_kepler} to 
\cref{fig:megno_example}  is the much larger volume of phase space 
occupied by chaotic orbits. This is due to the fact that the circumbinary
systems have an inner mass ratio larger by a factor of about 1000 compared
to a single-star system with Jupiter mass planets. The higher innner
mass ratio, combined with the fact that the stars in a circumbinary
system tend to be eccentric, leads to a formation of chaotic zone
extending all past the $5:1$ MMR for most binary eccentricities. 
This chaotic region is indeed unstable, \cite{Holman1999} ran full 
N-body simulations of test particles in close orbits around binaries
and later fit analytical formulue which give the size of the instability
region. Later, \cite{Mudryk2006} found that the instability region
is due to resonance overlap, as is visible in \cref{fig:megno_kepler}.

Looking at the analytic predictions for the resonant widths in 
\cref{fig:megno_kepler}, we see that all regions of resonance overlap
lead to chaotic trajectories. For all but one of the systems, the 
resonant widths roughly follow the first emerging chaotic trajectories.
The exception is Kepler-34 which orbits a highly eccentric binary
and is located in a region with mostly black dots. Since Kepler-34 
is a confirmed exoplanet with a well determined orbit, we conclude 
that the chaotic region in the first pane which does not overlap with 
the analytical prediction is most likely reasonably stable. All the 
other circumbinary planets lie comfortably outside the inner
chaotic zone and none of the planets lie within the $5:1$ MMR, which
is why the $6:1$ MMR is the first we considered in detail in 
\cref{ch:analytical_model}. Notice most importantly that
the extent of the inner chaotic region at moderete planet eccentricities
depends crucially on the binary eccentricity $e_i$, the higher it is, 
the further away an inner most planet has to be in order to be on a
stable orbit. \cite{Sutherland2015} investigate the chaotic zone
in detail by means of N-body simulations and find that the most likely
outcome for a planet in the inner chaotic zone is ejection, and a 
significant fraction of planets collider with one of the stars.
\begin{figure}[!t]
\centering
\includegraphics[width=\linewidth]{gfx/megno_kepler.pdf}
\caption{MEGNO maps for the observed circumbinary systems with relatively
    close-in planets. Red dots show observationally derived values for 
    the orbital parameters of the planets. The color coding shows the MEGNO 
    value, darker shades 
    corresponding to chaotic orbits with high MEGNO and lighter ones to 
    regular orbits. Plotted on top of the MEGNO maps is the analytical 
    prediction for the resonance widths of $n:1$ mean-motion resonances, 
    starting with $n=2$ (see \cref{ssub:Resonance_widths}). 
The dashed white line in
    the top left corner of each panel  is defined 
    by $R_p/a_i=1$ where $R_p=a_o(1-e_o)$
    is the outer periastron distance. Every outer orbit above that line
    is crossing the inner orbit and is therefore guaranteed to be 
    unstable in almost all cases.}
\label{fig:megno_kepler}
\end{figure}

\section{Coupling between \texttt{binary\_c} and REBOUND}
\label{sec:coupling_binary_c_rebound}
In order to realistically simulate the dynamical evoulution of a circumbinary
system, we need to combine a model for the orbital evolution of the stellar
binary given by \texttt{binary\_c} with the N-body code \texttt{REBOUND}. A best
conceivable solution would be an N-body code which at each timestep on top of
the gravitational forces acting on the three bodies also calculates the 
tidal forces acting on the binary. Since 
the tidal forces depend on all kinds of stellar parameters such as the
luminosities of the stars, their temperatures, core masses etc.; 
we would also need the full output of the \texttt{binary\_c} code between each 
N-body timestep. Actually, not only would we need to run \texttt{binary\_c} 
between each timestep in the N-body code but also between each \emph{substep}  
of a given timestep since the \texttt{IAS15} integrator requires several force
evaluations in an interval $[0,dt]$.

While not impossible, such approach would require a lot of work and and we
would need to make sure that \texttt{binary\_c} is able to work with timesteps
of a fraction of the dynamical time in the system (on the order of days), when
it normally works with timesteps comparable to stellar evolution timescales 
(millions of years). This would almost certainly be challenging and would most
likely
require some sort of interpolation scheme for the \texttt{binary\_c} output
variables.

There is  however an easier solution to the problem. We could ignore the gravitational
influence of the circumbinary planet on the stellar binary and in the N-body
code force the binary to move in a way consistent with the output of 
\texttt{binary\_c} generated beforehand. This would mean that we are in a regime
of a \emph{restricted three body problem} where the inner two masses do not 
experince the force of the third mass\footnote{Acutally, this is not strictly
true for the implementation in the code. The calculation of the forces on the 
stars at each \texttt{IAS15} timestep does include the influence of the planet.
However, since the stars are 'forced' to move according to a trajectory 
previously generated
beforehand with \texttt{binary\_c}, which was derived assuming an isolated system of 
two stars, it is still effectively a restricted problem over sufficiently long
timescales (at least one \texttt{binary\_c} period).}.
, a reasonable approximation because
the mass ratio between the total mass of the stellar binary in a circumbinary
system and the outer planet is $m_3/m_{12}\approx 10^{-3}$.

The question remaining then is how to 'force' the stellar binary in \texttt{REBOUND}
to move in the $(a,e)$ such that its trajectory matches the output from 
\texttt{binary\_c}. There are two options, the first one is to directly change 
the orbital elements $a_i$ and $e_i$ after each internal \texttt{binary\_c} timestep
(long compared to \texttt{IAS15} timestep). The other approach is to construct 
a force which acts on the stellar binary such that the values of $(a_i,e_i)$ after
specified time match the \text{binary\_c} values at that time; such force is 
necessarily velocity-dependent. Both approaches
are implented in \texttt{REBOUNDx}, a library for incorporating additional effects
in \texttt{REBOUND} simulations (\url{https:// github.com/dtamayo/reboundx}), as
functions \texttt{modify\_orbits\_direct} and \texttt{modify\_orbits\_forces}. We opt
to use the second approach because it is more physically motivated than directly 
changing the orbital elements 'by hand'. Indeed, a velocity-dependent force
 is exactly how the tidal decay of the binary happens in the first place.

In \texttt{REBOUNDx}, the force is constructed such that when orbit-avaraged it
gives an \emph{exponential} growth/decay of the semi-major axis and the
eccentricity. The approach is based on \cite{papaloizou}; the force governing 
the eccentricity evolution is given by
\begin{equation}
    \vect{a}_\text{damp}=-2 \frac{(\vect{v}\cdot\vect{r})\vect{r}}{r^2 \tau_e }
\end{equation}
and for the semi-major axis evolution
\begin{equation}
    \vect{a}_\text{damp}=- \frac{\vect{v}}{\tau_a} 
\end{equation}
where $\vect{a}$ is the acceleration vector. These two forces act in a way such
that the eccentricity/semi-major axis exponentially increase/decrease by a factor
of $e=2.71\dots$ over a timescale $\tau_e$/$\tau_a$. The eccentricity damping
keeps the angular momentum constant which meeans that some semi-major axis 
evolution is always induced. It also induces some pericentre precession. Both
effects are physical. 

The problem of changing the orbital elements of the stellar binary then reduces
to a suitable choice of two damping timescales, between each \texttt{binary\_c}
timesteps. An exponential decay in the semi-major axis is described by
the differential equation
\begin{equation}
    \left( \frac{\dot{a}}{a} \right)= \frac{1}{\tau_a} 
\end{equation}
which can be easily solved with the method of the separation of variables, giving
\begin{equation}
    \ln a_{k+1}-\ln a_k= \frac{t_{k+1}-t_k}{\tau_a} 
\end{equation}
where we have discretized the solution with $a_k$ designating the value of the
semi-major axis at timestep $k$ and $a_{k+1}$ the value at timestep $k+1$. The
damping timescale is then
\begin{equation}
    \tau_a = \frac{t_{k+1}-t_k}{\ln \frac{a_{k+1}}{a_k} } 
\end{equation}
Therefore, given the values of $a_k$ and $a_{k+1}$ (which are precalculated with
\texttt{binary\_c}), we can determine the exponential decay timescale and pass 
it to \texttt{REBOUND} such that ate time $t_{k+1}$ the value of $a$ in 
\texttt{REBOUND} matches the one in \texttt{binary\_c}. The algorithm for
running a simulation of a circumbinary system is then
\begin{enumerate}
    \item Choose stellar parameters, evolve binary system using \texttt{binary\_c}
        from a point at which significant changes in $(a,e)$ start occuring (RGB 
        branch) up to the start of the common envelope phase. Save output data
        to file.
    \item Load orbital parameters from \texttt{binary\_c}, calculate array of
        damping timescales $\tau_a$ and $\tau_e$ for each \texttt{binary\_c}
        timestep.
    \item Integrate system using \texttt{REBOUND} with the \texttt{IAS15} integrator
        from the zeroth to the final value of time in \texttt{binary\_c} output, at
        each time $t$ for which there is data from \texttt{binary\_c} change 
        damping timescales with a new value from array
\end{enumerate}
\begin{figure}[!t]
\centering
\includegraphics[width=\linewidth]{gfx/rebound_binary_c_coupling.pdf}
    \caption{Results of a \texttt{REBOUND} integration of a two-body system representing
    a stellar binary. The left panel shows the relative error in the orbital elements
    $a$ and $e$ between \texttt{REBOUND} and \texttt{binary\_c}. The right panel
    shows the output of \texttt{REBOUND} (blue dots) between two \texttt{binary\_c}
    timesteps (orange stars). The integration time in \text{REBOUND} has been
    rescaled by a factor of $10^3$ for reasons of computational efficiency.}
\label{fig:reb_bin_coupling}
\end{figure}
\Cref{fig:reb_bin_coupling}
shows the result of an N-body integration of a system with masses $M_1=1.4 M_\odot$,
$M_2=0.7 M_\odot$, initial eccentricity $e=0.3$ and initial period $P=200$ days. 
The integration time in REBOUND has been rescaled by a factor of $10^3$ compared to 
the output in \texttt{binary\_c} in order to reduce runtime; in the simulations which
include an outer planet the integration runs for the complete duration.
The left panel shows the relative error of the orbital elements calculated by
\texttt{REBOUND} compared to those from \texttt{binary\_c}. The right panel shows a 
small portion of the total integration time between two \texttt{binary\_c}
timesteps (orange stars). The errors are negligible. Even though this is a
two-body system, the energy is not conserved because of the presence of a 
velocity-dependent force (or equivalently, dissipative processes). 

\section{Intial conditions}
\label{sec:Intial conditions}
Before running full N-body simulations of circumbinary systems, a suitable choice
of initial conditions needs to be made. Since N-body simulations are in general very
computationally costly, we need to run simulations which explore the various parts
of the parameter space and which can be done in reasonable time on the available 
computer architecture. Unfortunately the N-body problem with small N is not 
parallelizable and is therefore largely limited by the CPU clock speed which has not
progressed significantly in the past decade. Having access to large clusters 
of CPUs is therefore only useful for running many simulations with different 
initial conditions.

The runtime of any given simulation depends crucially on the smallest dynamical 
timescale of interest. In this case, it is the orbital period of the stellar binary,
or rather, a fraction of it. We are most interested in circumbinary systems similar 
to the observed once on the main sequence, those have binary periods on the order of days.
Because the binary stars in such systems evolve slowly on timescales of tens to 
hundreds of millions of years from RGB to common-envelope, the simulations would 
require months of computer time. Instead we choose a fixed value of the initial period of
200 days which is approximately $~10$ times longer compared to observed systems. 
The \emph{shape}  of evolutionary tracks of binary stars in semi-major axis-eccentricity 
space only weakly dependends on the initial period (much more important are the masses)
and as long as the adiabatic criterion for resonant passage is satisfied in both cases,
its outcome will stay the same. Nevertheless, we do run a few simulations with
shorter period to check this is indeed the case.

The initial conditions are presented in \cref{table:initial_conditions}. For the 
primary mass we choose three values, ranging from $1.2\,M_\odot$ to $2.0\,M_\odot$.
Lower primary masses are not interesting because such binary stars take too long
to evolve past the common envelope phase and as a consequence we cannot observe
their final post common-envelope configurations. On the higher end, the case
$m1=2.0\,M_\odot$ is interesting because it marks the boundary between stars which
undergo helium flash ($\leq 2\,M_\odot$) and those that don't. The latter have
different trajectories in $(a,e)$ space with an abrupt decline in semi-major 
axis and eccentricity immediately before the onset of the common-envelope phase.
We specify the secondary mass with a mass ratio $q=m_1/m_2$ for three different 
values. The case of binaries with mass fraction near unity is not considered
because such systems undergo two common-envelope events in a short time span.

The inner period is fixed at 200 days as mentioned previously. For the 
eccentricity, perhaps the most important parameter for the global dynamics, 
we choose a circular value of $e=0.0$, a slightly eccentric value of $e=0.1$,
and a highly eccentric value of $e=0.3$. The outer period, corresponding
to the planet, is chosen such that initially, the planet is outside the 
inner chaotic zone. For $e_i=0.$ and $e_i=0.1$ we choose $P_o=5.8\,P_i$, which
is just inside the $6:1$ MMR, and for the highly eccentric binary with $e_i=0.3$,
we place the planet just inside the $7:1$ resonance because in that case the 
inner chaotic zone has a greater extent.
\begin{table}[h!]
\centering
\begin{tabular}{lc}
\toprule
    Parameter &  Value\\
\midrule
    $m_1\,[M_\odot]$ &1.2, 1.6, 2.0\\
    $q$&0.3, 0.6, 0.8\\
    $P_i$ [days]& 200\\
    $e_i$ &0.0, 0.1, 0.3\\
    $P_o/P_i$ & 5.8/6.8 \\
    $e_o$&$e_\text{forced}$\\
    $\varpi_i-\varpi_o$  & 0\\
         \bottomrule
\end{tabular}
    \caption{The initial choice of parameters for the N-body simulations.}
\label{table:initial_conditions}
\end{table}

The outer eccentricity is set to a fixed value of the secularly forced 
eccentricity for each system. To find the forced eccentricity, we 
run a short \txtttt{REBOUND} simulation without any coupling to 
\texttt{binary\_c}, taking $e_o=0$ for the initial value. After the 
simulation has finished, we calculate the centroid of points in 
$(e_o\cos\varpi_o,e_o\sin\varpi_o)$ plane, which is the forced eccentricity
(see \cref{sec:secular_evolution}). Finally, the difference in pericentres
is kept at a fixed value of zero, that is, the orbits are aligned. Starting
at a different value of $\varpi_i-\varpi_o$ would not change the results 
considerably
because the relative orientations of the two orbits become randomized on a short 
timescale.

\end{document}
